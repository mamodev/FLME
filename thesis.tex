\documentclass[a4paper, oneside, openright]{report}

\usepackage[a4paper,top=3cm,bottom=3cm,left=3cm,right=3cm]{geometry} 
\usepackage[fontsize=13pt]{scrextend}
\usepackage[english,italian]{babel}
\usepackage[fixlanguage]{babelbib}
\usepackage[utf8]{inputenc} 
\usepackage[T1]{fontenc}
\usepackage{lipsum}
\usepackage{rotating}
\usepackage{fancyhdr}               

\usepackage{amssymb}
\usepackage{amsmath}
\usepackage{amsthm}         

\usepackage{graphicx}
\usepackage[dvipsnames]{xcolor}         
\usepackage{listings}          
\usepackage{hyperref}     
\usepackage[normalem]{ulem}

\usepackage{titlesec}

\usepackage{lipsum}
\usepackage{setspace}
\usepackage{tocloft}

\usepackage{subfig}



\pagestyle{fancy}
\fancyhf{}
\lhead{\rightmark}
\rhead{\textbf{\thepage}}
\fancyfoot{}
\setlength{\headheight}{12.5pt}

\fancypagestyle{plain}{
  \fancyfoot{}
  \fancyhead{}
  \renewcommand{\headrulewidth}{0pt}
}

\let\oldsection\section
\renewcommand{\section}{\newpage\oldsection}

\makeatletter
\newcommand\footnoteref[1]{\protected@xdef\@thefnmark{\ref{#1}}\@footnotemark}
\makeatother

\lstdefinestyle{codeStyle}{
commentstyle=\color{teal},
keywordstyle=\color{Magenta},
numberstyle=\tiny\color{gray},
stringstyle=\color{violet},
basicstyle=\ttfamily\footnotesize,
breakatwhitespace=false,     
breaklines=true,                 
captionpos=b,                    
keepspaces=true,                 
numbers=left,                    
numbersep=5pt,                  
showspaces=false,                
showstringspaces=false,
showtabs=false,
tabsize=2
} \lstset{style=codeStyle}





\lstdefinestyle{longBlock}{
    commentstyle=\color{teal},
    keywordstyle=\color{Magenta},
    numberstyle=\tiny\color{gray},
    stringstyle=\color{violet},
    basicstyle=\ttfamily\scriptsize,
    breakatwhitespace=false,         
    breaklines=true,                 
    captionpos=b,                    
    keepspaces=true,                 
    numbers=left,                    
    numbersep=5pt,                  
    showspaces=false,                
    showstringspaces=false,
    showtabs=false,                  
    tabsize=2
} \lstset{style=codeStyle}


\lstset{aboveskip=20pt,belowskip=20pt}

\hypersetup{
    colorlinks,
    linkcolor=Black,
    citecolor=Black
}

\newtheorem{definition}{Definizione}[section]
\newtheorem{theorem}{Teorema}[section]
\providecommand*\definitionautorefname{Definizione}
\providecommand*\theoremautorefname{Teorema}
\providecommand*{\listingautorefname}{Listing}
\providecommand*\lstnumberautorefname{Linea}

\raggedbottom
\setlength{\cftbeforechapskip}{6pt}




\onehalfspacing

\title{A user-friendly framework for cross-device asynchronous Federated Learning}
\author{Marco Morozzi}
\graphicspath{ {./assets/} }


\begin{document}
\clearpage{}\newgeometry{top=3cm, bottom=3cm, left=3cm, right=3cm}

\begin{titlepage}
\begin{figure}[!htb]
    \centering
    \includegraphics[keepaspectratio=true,scale=0.5]{cherubinFrontespizio.eps}
\end{figure}

\begin{center}
    \LARGE{UNIVERSITÀ DI PISA}
\vspace{5mm}
    \\ \LARGE{Laurea Triennale in Informatica}
\end{center}

\vspace{10mm}
\begin{center}
    {\LARGE{\bf 
        A user-friendly framework for 
        cross-device asynchronous 
        Federated Learning  
    }}
\end{center}
\vspace{30mm}

\begin{minipage}[t]{0.47\textwidth}
	{\large{Relatore:}{\normalsize\vspace{3mm}
	\bf\\ \large{Prof: Massimo Torquati} \normalsize\vspace{3mm}\bf \\ \large{Prof: Patrizo Dazzi}}}
\end{minipage}
\hfill
\begin{minipage}[t]{0.47\textwidth}\raggedleft
	{\large{Candidato:}{\normalsize\vspace{3mm} \bf\\ \large{Marco Morozzi}}}
\end{minipage}

\vspace{30mm}
\hrulefill
\\\centering{\large{ANNO ACCADEMICO 2023/2024}}

\end{titlepage}
\restoregeometry
\clearpage{}
\tableofcontents
\newpage
\clearpage{}\chapter{Introduzione}
Si stima che nel mondo siano in uso oltre $4.6$\footnote{https://www.gsma.com/r/somic/} miliardi di smartphone e più di 30\footnote{https://www.unipoltech.com/it/news/scenario-iot} miliardi di dispositivi IoT connessi a livello globale, un numero in costante crescita \cite{Albreem2021Green}. I dispositivi IoT e mobile generano quotidianamente una mole straordinaria di dati, dai messaggi di testo ai parametri biometrici, dalle transazioni online alle abitudini di consumo, fino ai dati raccolti da sensori ambientali e wearable. Secondo l'International Data Corporation (IDC), il volume di dati generato dall'IoT a livello globale potrebbe raggiungere 180 zettabyte (ZB) entro il 20 \cite{Shi2019Edge}. Tali informazioni rappresentano un'enorme opportunità per il Machine Learning e per lo sviluppo di modelli predittivi in grado di trasformare i dati grezzi in conoscenza. Tuttavia, il paradigma tradizionale dell’apprendimento automatico richiede che i dati vengano raccolti, centralizzati ed elaborati in data center dedicati o su server cloud. Sebbene questo approccio sia stato ampiamente utilizzato, presenta criticità sempre più evidenti sia in termini di scalabilità che di rispetto della privacy.

\subsubsection*{Le sfide della centralizzazione dei dati}
La centralizzazione dei dati comporta due grandi problematiche. La prima riguarda la dimensione stessa dei dati, che richiede risorse di rete significative per il loro trasferimento. Con miliardi di dispositivi che generano dati in tempo reale, le infrastrutture necessarie per supportare questo flusso continuo di informazioni diventano estremamente costose sia da implementare che da mantenere. Inoltre, l'archiviazione centralizzata di tali volumi di dati, che richiede infrastrutture fisiche ed energetiche non sostenibile su larga scala, aggrava il problema dell'impatto ambientale ed aumenta significativamente i costi operativi per garantire la continuità dei servizi.

La seconda problematica, forse ancora più critica della prima, riguarda la privacy e la sicurezza dei dati. Trasferire grandi quantità di dati sensibili, come informazioni personali, dati sanitari o dettagli finanziari, verso un'unica entità centrale introduce rischi significativi. Violazioni della privacy, accessi non autorizzati, cyberattacchi e utilizzi impropri dei dati sono alcune delle principali minacce legate a questo approccio. Queste preoccupazioni sono accentuate dalla crescente attenzione delle normative internazionali, come il \textit{Regolamento Generale sulla Protezione dei Dati} (denominato GDPR) in Europa \cite{gdpr}, che impongono standard sempre più rigidi per la gestione e la protezione dei dati personali.

\subsubsection*{Un nuovo paradigma: il Federated Learning}
In risposta a queste sfide, il Federated Learning (FL) \cite{DBLP:journals/corr/McMahanMRA16} emerge come un paradigma innovativo che permette di superare i limiti del Machine Learning centralizzato. Questo approccio consente di spostare il calcolo verso i dati, anziché trasferire i dati verso il calcolo. In altre parole, i modelli vengono addestrati direttamente sui dispositivi locali, come smartphone, tablet o sensori IoT, utilizzando i dati presenti e raccolti sul dispositivo stesso. Solo i parametri aggiornati del modello vengono inviati a un server centrale per l’aggregazione, preservando così la privacy e riducendo drasticamente la quantità di dati trasmessi.

Il Federated Learning offre una risposta efficace alle problematiche di privacy e scalabilità. La possibilità di addestrare modelli direttamente sui dispositivi apre nuove prospettive per il learning automatico, specialmente in settori critici come la sanità, la finanza e le applicazioni consumer. Ad esempio, il completamento automatico delle parole sulla tastiera di smartphone (ad esempio Gboard di Google \cite{DBLP:journals/corr/abs-1811-03604}) o i sistemi di riconoscimento vocale negli assistenti virtuali (ad esempio Apple Siri \cite{DBLP:journals/corr/abs-2107-12603}) sono già oggi esempi concreti di Federated Learning applicato su larga scala.

\subsubsection*{Le sfide del Federated Learning}
Nonostante i vantaggi, implementare un sistema di Federated Learning efficiente, scalabile e sostenibile in contesti cross-device, ossia su reti altamente distribuite e composte da un numero elevato di dispositivi eterogenei, non è un compito banale. I dispositivi che partecipano a questi sistemi variano notevolmente in termini di risorse computazionali, capacità di memoria e connettività. Inoltre, i dati raccolti da questi dispositivi sono spesso non indipendenti e non identicamente distribuiti (non-i.i.d.), riflettendo le caratteristiche uniche degli utenti o dei contesti in cui vengono generati. Questa eterogeneità statistica complica la creazione di un modello globale che sia rappresentativo e performante per tutti i partecipanti \cite{li2020convergencefedavgnoniiddata}.

Anche dal punto di vista infrastrutturale, il Federated Learning pone sfide significative. La gestione di milioni di dispositivi implica la necessità di protocolli di comunicazione robusti e di algoritmi di aggregazione che possano operare in modo efficace nonostante la partecipazione dinamica e imprevedibile dei dispositivi, molti dei quali potrebbero disconnettersi durante l'addestramento. Questi aspetti rendono cruciale la progettazione di sistemi resilienti, capaci di tollerare latenze elevate nelle comunicazioni, nonché errori di comunicazione dei dati ed una partecipazione altamente dinamica delle entità della rete di comunicazione.

\subsubsection*{La motivazione e gli obiettivi della tesi}
Mentre il Federated Learning si afferma come una tecnologia promettente, l'implementazione di soluzioni pratiche che scalino a reti cross-device richiede competenze tecniche avanzate e trasversali, spesso lontane dal focus principale di un esperto di Machine Learning. La progettazione di un sistema di Federated Learning implica infatti la risoluzione di problemi legati all'architettura del sistema, alla gestione delle comunicazioni, alla sincronizzazione dei dispositivi e all’ottimizzazione degli algoritmi di aggregazione. Questo può rappresentare un ostacolo significativo per i ML Model Developers e Data Scientist il cui obiettivo principale in questo contesto è progettare e addestrare modelli per casi d'uso specifici.

Questa tesi nasce proprio da queste considerazioni. L'obiettivo è sviluppare un framework user-friendly che renda l'adozione del Federated Learning accessibile anche agli sviluppatori non esperti in infrastrutture distribuite. Il framework proposto mira a semplificare il processo di implementazione, fornendo strumenti che permettano agli sviluppatori di concentrarsi prevalentemente sulla definizione del modello e sulla raccolta dei dati, senza doversi preoccupare delle complessità tecniche sottostanti.

Inoltre, il framework è progettato per affrontare le sfide specifiche del contesto cross-device, integrando soluzioni per la gestione dell'eterogeneità dei dispositivi, la variabilità statistica dei dati e le limitazioni delle risorse computazionali. L'obiettivo finale è offrire un ambiente di sviluppo che sia trasparente per l'utente, replicando l'esperienza di un sistema centralizzato ma beneficiando delle caratteristiche distribuite e di privacy-preserving del Federated Learning.

\subsubsection*{Struttura della tesi}
Nei capitoli successivi, verranno introdotti i principi fondamentali del Machine Learning e del Federated Learning, con un focus particolare sulle sfide che caratterizzano gli scenari cross-device. Saranno analizzati i principali approcci per gestire l'eterogeneità dei dispositivi e dei dati, così come le architetture software che supportano l'apprendimento federato. In seguito verranno discusse le soluzioni attualmente presenti sul mercato ed introdotta la struttura del framework elaborato oggetto di questa tesi descrivendone le caratteristiche tecniche, le innovazioni introdotte e i risultati sperimentali ottenuti, evidenziando il suo contributo nel rendere il Federated Learning più accessibile, scalabile ed efficace. 

\subsubsection*{Codice del framework}
Il codice sorgente del framework disegnato e sviluppato in questa tesi è pubblico e scaricabile da Github al seguente link: \\
\href{https://github.com/mamodev/Async-Federated-Learnig}{https://github.com/mamodev/Async-Federated-Learnig}
\clearpage{}
\clearpage{}\chapter{Background}\label{background}

In questo capitolo vengono presentate le informazioni e le conoscenze di base necessarie per comprendere il contesto del Federated Learning, con particolare attenzione al suo utilizzo in ambienti cross-device e alle sfide che esso comporta. Verranno introdotti i principi fondamentali dell’apprendimento federato, un paradigma di Machine Learning progettato per preservare la privacy dei dati distribuendo il processo di addestramento su diversi dispositivi. Saranno inoltre esplorati i concetti chiave di eterogeneità statistica, comunicazione asincrona e gestione dei Client, essenziali per comprendere le complessità di un sistema di Federated Learning cross-device.

Questa panoramica fornisce al lettore una solida base per apprezzare il contributo tecnico e metodologico del lavoro svolto in questa tesi, che mira a sviluppare un framework user-friendly per implementare algoritmi di Federated Learning asincroni in contesti cross-device, affrontando le sfide di scalabilità, efficienza e semplicità d’uso per l'utente.



\section{Machine Learning}
Il Machine Learning (ML) è una disciplina dell'intelligenza artificiale che si concentra sullo sviluppo di algoritmi e modelli statistici capaci di apprendere e migliorare le proprie prestazioni attraverso l'esperienza, senza essere esplicitamente programmati per svolgere un compito specifico. In pratica, il ML consente ai computer di identificare pattern e inferire regole dai dati, permettendo loro di fare previsioni o prendere decisioni basate su nuovi input.\\ \\
\noindent I modelli di Machine Learning vengono addestrati utilizzando insiemi di dati, che possono contenere esempi etichettati o non etichettati. A seconda del tipo di apprendimento e della natura dei dati, il ML si suddivide principalmente in tre categorie:
\begin{itemize}
    \item \textbf{Apprendimento Supervisionato}: Il modello viene addestrato su un dataset etichettato, dove ogni esempio di input è associato a un output desiderato. L'obiettivo è apprendere una funzione che mappa gli input agli output corretti, permettendo al modello di fare previsioni accurate su dati nuovi e non visti.
    
    \item \textbf{Apprendimento Non Supervisionato}: Il modello lavora con dati non etichettati e cerca di identificare strutture nascoste o pattern all'interno dei dati. Questo tipo di apprendimento è utilizzato per compiti come il clustering, dove gli esempi vengono raggruppati in base a somiglianze intrinseche.

    \item \textbf{Apprendimento per Rinforzo}: Il modello, spesso chiamato agente, interagisce con un ambiente dinamico e apprende a compiere azioni che massimizzano una ricompensa cumulativa. L'agente prende decisioni sequenziali, adattandosi in base al feedback ricevuto dalle sue azioni precedenti.
\end{itemize}

Per lo scopo di questa tesi ci concentriamo su ML supervisionato il quale può essere descritto attraverso le seguenti fasi:
\begin{itemize} 
    \item \textbf{Raccolta dei Dati}: I dati vengono acquisiti da diverse fonti, come dispositivi mobili, sensori, applicazioni web o database aziendali. Questi dati possono includere informazioni strutturate o non strutturate, e spesso contengono dati sensibili o personali.
   
    \item \textbf{Trasferimento e Centralizzazione dei Dati}: I dati raccolti vengono trasferiti attraverso reti di comunicazione al server centrale. Questo processo può comportare l'invio di grandi volumi di dati, richiedendo una larghezza di banda significativa e introducendo potenziali rischi per la sicurezza durante il trasferimento.
    
    \item \textbf{Addestramento del Modello}: Il modello di Machine Learning viene addestrato utilizzando i dati pre-elaborati. Questo coinvolge l'uso di algoritmi di ottimizzazione per minimizzare una funzione di perdita, adattando i parametri del modello in modo da migliorare la sua capacità di fare previsioni accurate.
    
    \item \textbf{Validazione e Test}: Il modello addestrato viene valutato su un set di dati di validazione per misurare le sue prestazioni e prevenire problemi come l'overfitting. Se necessario, si iterano processi di tuning degli iperparametri o si esplorano architetture alternative per migliorare i risultati.
    
    \item \textbf{Implementazione e Distribuzione}: Una volta che il modello soddisfa i criteri di prestazione desiderati, viene implementato in un ambiente di produzione. Questo può includere l'integrazione in applicazioni software, servizi web o sistemi embedded, dove il modello fornisce previsioni o prende decisioni in tempo reale.
\end{itemize}

Il Machine Learning ha rivoluzionato numerosi settori, tra cui il riconoscimento vocale, l'elaborazione del linguaggio naturale, la visione artificiale e l'analisi predittiva. Tuttavia, l'approccio tradizionale di addestramento prevede la raccolta e la centralizzazione dei dati provenienti da varie fonti, un approccio che che può comportare rischi significativi in termini di privacy, sicurezza e conformità normativa. Questi rischi emergono dal trasferimento e dall'archiviazione di dati potenzialmente sensibili in un unico luogo, esponendoli a una maggiore vulnerabilità rispetto ad accessi non autorizzati e a possibili violazioni della privacy.

\section{Federated Learning}
Il Federated Learning (FL) è una tecnica innovativa di machine learning supervisionato che consente l'addestramento di modelli di intelligenza artificiale su dati che rimangono distribuiti su diverse fonti, senza la necessità di centralizzare i dati in un unico punto. Esso inverte il paradigma classico del machine learning centralizzato spostando l'addestramento dei modelli dove si trovano i dati invece di spostare i dati dove avviene l'addestramento dei modelli. 
Nel machine learning centralizzato, i dati provenienti da diverse fonti vengono raccolti e trasferiti a un server centrale o a un data center. Qui, i dati vengono elaborati e utilizzati per addestrare il modello di machine learning.Nel Federated Learning (Apprendimento Federato), il calcolo viene spostato verso i dati. Invece di trasferire i dati a un server centrale, l'addestramento del modello avviene localmente sui dispositivi che possiedono i dati (esempi di dispositivi sono smarphone, desktop e sensori con capacità di calcolo). Solo i parametri del modello aggiornati vengono inviati a un server centrale, dove vengono aggregati per aggiornare il modello globale, garantendo che i dati personali non vengano mai esposti o trasferiti.

L'apprendimento federato è particolarmente importante in contesti dove la privacy e la sicurezza dei dati sono cruciali, come nel settore sanitario o finanziario. Inoltre, consente di sfruttare la potenza di calcolo distribuita, riducendo la necessità di infrastrutture centralizzate costose e complesse, nonché ridurre i requisiti di banda necessari per la trasmissione di grandi dataset in un unico centro di elaborazione. 

In modo esemplificativo possiamo affermare che: 
\begin{itemize}
    \item Il Machine Learning centralizzato prevede lo spostamento dei dati verso il calcolo
    \item Il Federated (Machine) Learning prevede lo spostamento del calcolo verso i dati.
\end{itemize}



\begin{figure}[h]
\centering
\includegraphics[width=14cm]{fed_overview}
\caption{Architettura logica dell'apprendimento federato. I modelli indicati con \textit{Model 1},...,\textit{Model N} sono addestrati localmente su dispositivi differenti ed accedendo esclusivamente a dati locali accessibili al dispositivo. I dati ottenuti dall'addestramento dei modelli locali (ad esempio parametri o pesi) vengono inviati ad un modello aggregato centrale (Aggregated Model) nel Cloud che potrà essere ritrasmesso come aggiornamento del modello locale ai dispositivi. \label{fig:architetturaFL}}
\end{figure}

In Figura~\ref{fig:architetturaFL} è schematizzata l'architettura logica dell'approccio all'apprendimento federato. Questo modello architetturale è stato introdotto da Google nel 2016 per preservare la privacy degli utenti e ridurre la necessità di trasferimento di grandi quantità di dati sensibili \cite{DBLP:journals/corr/McMahanMRA16}.

\subsubsection*{Cross-Silo vs Cross-Device}

Il termine Federated Learning è stato inizialmente introdotto con un'enfasi particolare sulle applicazioni per dispositivi mobili ed Edge, quindi a larghissima scala. Il termine ``Edge'' si riferisce a dispositivi o nodi di calcolo posizionati ai margini della rete, vicini alle fonti dei dati o agli utenti finali. Questi includono smartphone, sensori IoT e piccoli server locali che eseguono elaborazioni direttamente sul posto, riducendo così la latenza e il carico sulla rete centrale. A differenza dell'architettura Cloud tradizionale, dove i dati vengono inviati a data center centralizzati per l'elaborazione e l'archiviazione, l'architettura Edge sposta parte del carico computazionale verso i dispositivi periferici. Questo permette un'elaborazione più rapida e un minore utilizzo della larghezza di banda, poiché i dati non devono essere trasmessi interamente al Cloud.


Tuttavia, l'interesse per l'applicazione del Federated Learning si è notevolmente ampliato, includendo anche scenari che coinvolgono un numero limitato di Clienti relativamente affidabili. Un esempio di questi scenari è la collaborazione tra più organizzazioni distinte che vogliono cooperare per addestrare un modello senza però condividere i dati di ciascuna organizzazione.
Queste due distinte impostazioni di Federated Learning hanno dato vita a due modelli architetturali distinti denominati  \textit{Cross-Silo} e \textit{Cross-Device}.

\paragraph{Cross-Silo Federated Learning.} Questo modello architetturale di FL coinvolge organizzazioni, aziende o gruppi di clienti. In questo scenario, il numero di partecipanti è solitamente ridotto (ad esempio, qualche centinaio), ma ogni entità locale possiede una quantità significativa di dati locali. Un esempio pratico di Cross-Silo Federated Learning è la collaborazione tra istituti medici per addestrare modelli di previsione delle malattie utilizzando dati sensibili dei pazienti di ogni centro. 
In questo contesto in in cui il numero di partecipanti nell'addestramento del modello è relativamente piccolo e tipicamente corrisponde ad organizzative come aziende o istituzioni, ciascuna ha a disposizione grandi volumi di dati e grandi risorse computazionali affidabili. In questo scenario, la partecipazione delle entità all'apprendimento federato è più stabile e coordinata, grazie anche a connessioni di rete affidabili e ad alta velocità.

\paragraph{Cross-Device Federated Learning.} Questo modello coinvolge dispositivi mobili come smartphone, wearable sensors e dispositivi IoT. In questo caso, il numero di Client può raggiungere valori estremamente grandi (centinaia di migliaia o milioni), ma ogni Client ha una quantità relativamente piccola di dati locali. Questo approccio richiede la partecipazione di un gran numero di dispositivi edge per avere successo. Questa configurazione presenta alcune sfide peculiari, come la partecipazione dinamica e non prevedibile  dei dispositivi, le risorse limitate (in termini di capacità computazionale, memoria e batteria), e la variabilità delle connessioni che possono essere instabili.

In sintesi, il Cross-Silo Federated Learning è adatto per scenari in cui poche organizzazioni collaborano utilizzando grandi quantità di dati e risorse di calcolo significative, mentre il Cross-Device Federated Learning è tipico di scenari in cui molti (o moltissimi) dispositivi con piccole quantità di dati e potenza di calcolo relativa collaborano per addestrare un modello globale.

Per lo scopo di questo tesi ci limiteremo ad analizzare principalmente il contesto cross-device che presenta sfide ed opportunità maggiori.

\subsubsection*{Esempi di Applicazione dell'Apprendimento Federato}

Il Federated Learning ha trovato applicazione in diversi ambiti, dimostrando la sua efficacia nel preservare la privacy dei dati e nel migliorare l’efficienza dei modelli su dispositivi decentralizzati. Un esempio rilevante è il completamento automatico della tastiera sui dispositivi mobili, come implementato da Google nel suo servizio Gboard \cite{lian2017decentralizedalgorithmsoutperformcentralized}. In questo caso, i modelli apprendono dai dati di digitazione degli utenti direttamente sui loro dispositivi, evitando il trasferimento di informazioni sensibili verso server centrali e garantendo così la privacy.

Un altro esempio riguarda gli assistenti virtuali integrati negli smartphone, che utilizzano il Federated Learning per perfezionare i sistemi di riconoscimento vocale. Grazie a questo approccio, i dispositivi possono apprendere dai comandi vocali degli utenti, migliorando le capacità di comprensione senza trasmettere registrazioni vocali ai server, riducendo al minimo i rischi per la riservatezza.

In ambito sanitario, i dispositivi indossabili (wearable devices) raccolgono dati relativi alla salute, come il battito cardiaco, la pressione arteriosa e l’attività fisica che si sta svolgendo. Attraverso il Federated Learning, questi dispositivi possono addestrare modelli predittivi per monitorare le condizioni di salute degli utenti direttamente sui dispositivi stessi, senza trasferire dati sensibili al Cloud di riferimento dell'applicazione. In questo modo, si garantisce la riservatezza delle informazioni personali, preservando la privacy degli utenti.

In tutte queste applicazioni, è possibile riconoscere un insieme comune di passaggi necessari per definire il modello di apprendimento federato. A titolo esemplificativo, i principali passaggi sono elencati di seguito:
    \begin{itemize}
        \item \textbf{Inizializzazione del modello globale}: Si inizia con l'inizializzazione del modello globale (e dei suoi pesi) sul server centrale (denominato Master). Il modello può essere inizializzato casualmente o basato su un modello precedentemente addestrato.
    
        \item \textbf{Selezione dei Client}: Il Master seleziona, con una qualche politica, un sottoinsieme delle entità periferiche (denominate Client) che devono partecipare ad un round di addestramento.
    
        \item \textbf{Fornire il modello globale ai Client}: 
        Vengono inviati ai Client i parametri dell'ultima versione del modello, insieme ai parametri necessari per la fase di addestramento, come il numero di epoche, la dimensione dei batch, il learning rate, ecc.
    
        \item \textbf{Allenamento locale}: Ogni Client addestra il modello ricevuto dal Master utilizzando i propri dati locali. Al termine dell'addestramento, il Client invia al Master il modello locale aggiornato ed eventualmente anche alcuni metadati non sensibili dal punti di vista della privacy come la dimensione del dataset utilizzato.
        
        \item \textbf{Aggregazione}: 
         Una volta che ciascun Client selezionato ha inviato i risultati dell'addestramento, il Master combina tutti i modelli locali per creare un nuovo modello globale. L'aggregazione dei modelli locali può seguire diverse strategie e algoritmi; il più semplice è chiamato \textit{Federated Averaging} (FedAvg) \cite{DBLP:journals/corr/McMahanMRA16}, che calcola la media ponderata dei modelli locali utilizzando come peso la dimensione del dataset locale di ciascun Client.

        \item \textbf{Controllo di convergenza}: Il processo di apprendimento è un processo iterativo, che viene ripetuto fino al raggiungimento della convergenza del modello globale.
        
    \end{itemize}


\subsubsection*{Sfide dell'apprendimento federato} 

Sebbene il Federated Learning presenti numerosi vantaggi rispetto al Machine Learning centralizzato, esso non è privo di problematiche e sfide significative che devono essere affrontate e risolte. 

L'\textit{eterogeneità dei sistemi e dispositivi} che partecipano alla definizione del modello aggregato, e l'\textit{eterogeneità statistica} dei dati introducono problemi che hanno un impatto non trascurabile. Per  \textbf{eterogeneità dei sistemi} si intende la variabilità delle caratteristiche hardware e di connettività tra i dispositivi. Alcuni dispositivi potrebbero avere risorse molto limitate (e/o limitazione sul consumo energetico) e quindi potrebbero non essere in grado di sostenere il training di reti complesse. Per \textbf{eterogeneità statistica} si riferisce alla variazione e alla distribuzione non uniforme dei dati tra i dispositivi partecipanti. In un contesto distribuito su vari dispositivi i dati riflettono le attività e le caratteristiche specifiche degli utenti o dei dispositivi, introducendo differenze significative nelle statistiche dei dati locali. Questa variabilità statistica può manifestarsi in vari modi:

\begin{itemize}
\item \textit{Distribuzione dei dati non i.i.d.}: i dati locali su ciascun dispositivo non seguono la stessa distribuzione (non sono \``indipendenti e identicamente distribuiti\''). 
\item \textit{Disparità nella quantità di dati}: ogni dispositivo potrebbe avere una quantità diversa di dati locali a disposizione per l'addestramento. 
\item \textit{Differenze nei pattern dei dati}: i dati locali possono contenere pattern diversi a seconda del contesto in cui vengono generati sulla base di fattori come la regione geografica, le preferenze personali e il comportamento.
\end{itemize}

L'eterogeneità dei sistemi e statistica pone sfide significative per il Federated Learning, poiché rende difficile creare un modello globale che funzioni bene per tutti i Client che partecipano all'addestramento. Algoritmi specifici, come FedAvg~\cite{DBLP:journals/corr/McMahanMRA16}, devono essere adattati per gestire questa variabilità e garantire che l'aggregazione dei modelli locali porti a un modello globale efficace, pur rispettando le caratteristiche uniche di ciascun Client. Inoltre possono introdurre aspetti come: 
\begin{itemize}
    \item  \textbf{Convergenza rallentata}:
    la disomogeneità dei dati rallenta la convergenza degli algoritmi di Federated Learning. \cite{li2020convergencefedavgnoniiddata}

    \item \textbf{Incoerenza dell'obiettivo}: l'aggregazione di modelli addestrati su dati non i.i.d. può portare alla convergenza verso un punto stazionario di una funzione obiettivo diversa da quella reale. \cite{DBLP:journals/corr/abs-2007-07481}
    
    \item \textbf{Instabilità del modello}: la differenza tra i modelli locali e quello globale, dovuta ai dati non i.i.d., può portare a una convergenza instabile che può non convergere mai alla funzione obiettivo globale. \cite{DBLP:journals/corr/abs-1910-06378}
\end{itemize}


\section{Eterogeneità dei Sistemi di Calcolo}\label{chap:sys-eter}

In~\cite{DBLP:journals/corr/McMahanMRA16}, è stato introdotto l'algoritmo Federated Averaging (FedAvg), che permette ai Client di aggiornare i propri modelli locali prima di inviare i parametri aggiornati al Server centrale per l'aggregazione. FedAvg prevede che ciascun Client esegua $E$ epoche (iterazioni sul proprio dataset locale) utilizzando lo Stochastic Gradient Descent (SGD) \cite{10.1214/aoms/1177729586} con una dimensione di mini-batch pari a $B$. Di conseguenza, per un Client con $n_i$ campioni di dati locali, il numero di iterazioni locali di SGD è calcolato come $t_i = E \cdot n_i / B$, valore che può variare notevolmente tra i Client.

Questa variazione è causata dall'eterogeneità dei sistemi di calcolo, un aspetto intrinseco del Federated Learning in cui i Client dispongono di risorse computazionali e dataset locali di dimensioni diverse. In particolare, dispositivi come smartphone, tablet e laptop differiscono in termini di capacità di elaborazione, memoria disponibile e durata della batteria, il che influenza la quantità di lavoro computazionale che ciascun Client può gestire. Inoltre, anche a parità di risorse di calcolo, rallentamenti possono essere causati da processi che girano in background, interruzioni, limitazioni di memoria. 
In generale, un dispositivo con un dataset locale più ampio ($n_i$ elevato) richiederà più iterazioni per completare $E$ epoche, mentre un dispositivo con risorse limitate potrebbe incontrare difficoltà anche solo ad eseguire poche iterazioni tra quelle previste. 

L'eterogeneità tra i Client comporta anche una variabilità significativa nelle prestazioni di addestramento locale, influenzando sia la velocità di convergenza che la qualità del modello aggregato. Per affrontare questa sfida, FedAvg e altri algoritmi di Federated Learning devono essere progettati per adattarsi a queste differenze, ad esempio, regolando dinamicamente il numero di epoche o il batch size in base alle risorse del dispositivo. Questo adattamento mira a garantire che anche i Client con risorse limitate possano contribuire all'aggiornamento del modello globale, mantenendo l'efficacia e l'efficienza dell' addestramento distribuito su una rete eterogenea di dispositivi.

La durata di ogni round è dunque determinata dai Client più lenti (strugglers). In un contesto dove il numero di Client è elevato  e le variazioni nella potenza computazionale e nel numero di iterazioni locali dei dispositivi sono significative, questo approccio non è ottimale ed impedisce al sistema di scalare. In una configurazione Cross-Device, inoltre, i dispositivi sono per definizione inaffidabili e potrebbero andare offline in qualsiasi momento.

Il problema principale è che FedAvg non permette ai partecipanti di eseguire quantità variabili di lavoro locale in base ai vincoli dei loro sistemi sottostanti (è comune semplicemente eliminare i dispositivi che non riescono a calcolare le epoche $E$ entro una finestra temporale specificata) \cite{DBLP:journals/corr/abs-1902-01046}.

Nella configurazione classica di FedAvg, in cui non viene permesso ai Client di eseguire una quantità variabile di lavoro e in cui non è permessa l'aggregazione di risultati parziali, la convergenza del modello è garantita \cite{DBLP:journals/corr/McMahanMRA16}. Queste assunzioni non sono però ragionevoli in un contesto Cross-Device. Nel momento in cui vengono rilassate queste assunzioni, il rischio è l'inconsistenza dell'obiettivo, cioè il modello globale potrebbe convergere ad un punto stazionario di una funzione obiettivo non corrispondente, che può essere arbitrariamente diversa dall'obiettivo reale (come mostrato in Figura~\ref{fig:objective_inconsistency}), oppure il modello potrebbe non convergere affatto e causare aggiornamenti instabili.

\begin{figure}[h]
\centering
\includegraphics[width=1.0\textwidth]{objective_inconsistency.png}
\caption{Effetto di addestramenti locali con numero di epoche ed iperparamentri omogenei ed eterogenei. I quadrati verdi e i triangoli blu denotano i minimi degli obiettivi globali e locali, rispettivamente.} \label{fig:objective_inconsistency}
\end{figure}

Nel FL la funzione obiettivo globale è definita come segue: $F(x) = \sum_{i=0}^m{n_i \cdot F_i(x)/n}$, dove $m$ è il numero di client e $n=\sum^m{n_i}$ il numero totale di campioni. Come dimostrato in \cite{DBLP:journals/corr/abs-2007-07481} la media standard dei modelli dopo aggiornamenti locali eterogenei (di conseguenza FedAvg) porta alla convergenza verso un punto stazionario, che non appartiene alla funzione obiettivo originale $F(x)$, ma ad una funzione obiettivo incoerente $F_e(x)$, che può essere arbitrariamente diversa da $F(x)$ a seconda dei valori $\tau_i$ (numero di epoche locali).
\newline
\newline
Per ottenere un'intuizione su questo fenomeno, si osservi la Figura~\ref{fig:objective_inconsistency} che, se il Client 1 ($x_1$) esegue più aggiornamenti locali rispetto al Client 2 ($x_2$), allora l'aggiornamento $x^{(t+1,0)}$ si allontana dal minimo globale vero $x^*$, dirigendosi verso il minimo locale $x_1^*$


\section{Eterogeneità Statistica dei Dati}
L'eterogeneità statistica dei dati determina in gran parte l'efficacia e l'efficienza del processo di apprendimento federato. In un contesto Cross-Device realistico i dati di addestramento sono distribuiti su un ampio numero di dispositivi Client, ognuno dei quali raccoglie dati in modo indipendente e con caratteristiche uniche, spesso non identicamente distribuiti e non indipendenti.

Come anticipato nella precedente sezione, FedAvg ha dimostrato successo empirico in configurazioni eterogenee purché vengano mantenute le sue assunzioni stringenti. In caso in cui esse vengono a mancare metodi come FedAvg hanno dimostrato di divergere in pratica quando i dati non sono i.i.d. \cite{DBLP:journals/corr/abs-1812-06127}.

L'eterogeneità statistica influisce negativamente sulla convergenza degli algoritmi di apprendimento federato e sulla qualità del modello globale. Quando i dati dei Client sono non-i.i.d., le direzioni di aggiornamento dei modelli locali possono essere molto diverse tra loro, causando oscillazioni nella fase di aggregazione e rallentando la convergenza \cite{DBLP:journals/corr/abs-1806-00582}. Inoltre, il modello globale potrebbe non rappresentare adeguatamente i modelli locali ottimali, riducendo le prestazioni complessive.

Un altro problema legato all'eterogeneità dei dati è la \textit{fairness}, poiché il modello appreso potrebbe mostrare un bias verso dispositivi con una maggiore quantità di dati o, se si ponderano i dispositivi in modo uguale, verso gruppi di dispositivi che compaiono più frequentemente.

La considerazione sulla scelta tra un singolo modello globale e approcci multi-modello nel Federated Learning è particolarmente rilevante, soprattutto quando i dati locali sui dispositivi sono non-i.i.d. Se ogni dispositivo è in grado di eseguire l'addestramento sui propri dati locali, come richiesto dal Federated Learning, sorge naturalmente la domanda: è sempre vantaggioso puntare a un unico modello globale?
Un singolo modello globale offre indubbiamente alcuni vantaggi. Ad esempio, può essere distribuito ai Client privi di dati o con dati limitati, garantendo una soluzione standardizzata e uniforme per tutti i partecipanti. Questo approccio è inoltre utile quando le caratteristiche dei dati dei dispositivi sono abbastanza simili, o quando la variabilità statistica è minima, rendendo il modello generale e sufficientemente efficace per tutti.

Tuttavia, nei casi in cui i dati sono fortemente non-i.i.d., un unico modello globale può non essere ideale, poiché potrebbe non riuscire a catturare le peculiarità dei dati locali di ciascun Client, con conseguente degrado delle prestazioni complessive. In questi scenari, approcci ``multi-modello'' come il Multi-Task Learning (MTL) e il Meta-Learning possono essere più appropriati ed efficaci~\cite{DBLP:journals/corr/abs-1912-04977}.

\begin{itemize}
\item \textbf{Multi-Task Learning (MTL)}: Con MTL, ogni dispositivo addestra un modello che, pur condividendo parte della struttura con il modello globale, è adattato specificamente alle caratteristiche dei dati locali. In sostanza, MTL tratta ogni dataset locale come un task distinto, consentendo la creazione di modelli specifici per ciascun Client. Questo approccio permette a ciascun dispositivo di ottimizzare il modello per il proprio contesto, mantenendo una connessione con il modello globale ma permettendo una maggiore flessibilità e migliorando così le prestazioni quando i dati sono eterogenei. 
\item \textbf{Meta-Learning}: Meta-Learning mira a creare un modello globale che non è semplicemente statico, ma è progettato per adattarsi rapidamente ai nuovi dati. In Federated Learning, un modello basato su Meta-Learning può apprendere una struttura globale e generica che ogni dispositivo può ulteriormente personalizzare sui propri dati, migliorando l’efficacia del modello per scenari con variabilità significativa.
\end{itemize}

Inoltre, la personalizzazione tramite tecniche come il fine-tuning locale o il meta-learning permettono di adattare un modello globale ai dati di ogni Client, garantendo una maggiore precisione a livello locale. 


Infine, l'eterogeneità statistica può avere effetti significativi anche sulla privacy e sulla sicurezza nel Federated Learning. Quando i dati locali dei dispositivi sono non-i.i.d. e altamente specifici, l'aggiornamento dei modelli locali può riflettere particolari dettagli distintivi dei dati di un Client. Questo può aumentare la vulnerabilità a potenziali attacchi, come gli \textit{attacchi di inversione del gradiente} \cite{zhang2022surveygradientinversionattacks}. In questi attacchi, un avversario potrebbe utilizzare le informazioni contenute negli aggiornamenti del modello per risalire a dati sensibili originari, sfruttando l’unicità dei pattern nei dati locali.
Per mitigare questi rischi, tecniche di protezione avanzate come la \textit{privacy differenziale} possono essere impiegate. La privacy differenziale aggiunge rumore controllato agli aggiornamenti dei modelli, rendendo più difficile per un attaccante inferire informazioni specifiche dai dati originali di un Client senza compromettere troppo la qualità dell'apprendimento globale. Questo approccio consente di proteggere le informazioni sensibili, bilanciando privacy e accuratezza nel Federated Learning~\cite{9714350}.


\section{Architetture Software di Riferimento}\label{chap:background:architetture}

Nel contesto del Federated Learning, l'architettura del sistema gioca un ruolo cruciale nel determinare l'efficienza, la scalabilità e la sicurezza del processo di addestramento distribuito. Esistono diverse architetture proposte in letteratura per gestire i vari requisiti e per affrontare le sfide del Federated Learning, ciascuna con i propri vantaggi e svantaggi~\cite{9359305}. Nel seguito descriviamo i quattro tipi principali di architettura software.


\begin{figure}
    \centering
    \subfloat{{\includegraphics[width=0.45\textwidth]{assets/arch-centralized.png} }} 
    \subfloat{{\includegraphics[width=0.45\textwidth]{assets/arch-hierarchical.png} }} \\
    
   \subfloat{{\includegraphics[width=0.45\textwidth]{assets/arch-regional.png} }}
   \subfloat{{\includegraphics[width=0.45\textwidth]{assets/arch-decentralized.png} }}
    
    \caption{Schema astratto della comunicazione del processo federato nelle diverse architetture}
    \label{fig:arch-types}
    
\end{figure}


\paragraph{Architettura centralizzata}

L'architettura software prevalente nei sistemi di Federated Learning attuali è quella centralizzata, in cui un singolo nodo Master coordina l'intera rete di dispositivi periferici come mostrato in Figura~
\ref{fig:arch-types}a. In questa configurazione, il Master comunica con tutti i client, raccoglie i modelli locali, esegue l'aggregazione e distribuisce il modello globale aggiornato a tutti i dispositivi. Questa architettura è particolarmente adatta a sistemi di piccole dimensioni, in quanto semplice da implementare e gestire.

Tuttavia, l'architettura centralizzata presenta alcune limitazioni significative:
\begin{itemize}
\item \textit{Singolo punto di fallimento}. Il Master è un elemento centrale che, per definizione, rappresenta un potenziale punto di vulnerabilità. Se il Master subisce un'interruzione, l’intero processo di apprendimento viene compromesso. Anche se organizzazioni o grandi aziende possono offrire server centrali affidabili per alcuni scenari, avere un Master sempre disponibile e potente potrebbe non essere fattibile o desiderabile in contesti più collaborativi, distribuiti o dinamici \cite{DBLP:journals/corr/VanhaesebrouckB16}.
\item \textit{Collo di bottiglia con molti Client}. Quando il numero di Client è elevato, il server Master potrebbe diventare un collo di bottiglia, riducendo l'efficienza e rallentando il processo di aggregazione. Lian et al. \cite{lian2017decentralizedalgorithmsoutperformcentralized}.
 hanno dimostrato che algoritmi decentralizzati possono superare i sistemi centralizzati in termini di prestazioni in scenari con numerosi client.
 \end{itemize}

Va rilevato comunque che con un’attenta progettazione del sistema è possibile mitigare l’impatto di un numero elevato di Client anche in un sistema centralizzato. Ad esempio, tecniche di ottimizzazione della comunicazione e gestione delle risorse possono migliorare la scalabilità del sistema, riducendo la latenza e distribuendo il carico in modo più efficiente \cite{DBLP:journals/corr/abs-1902-01046}.

\paragraph{Architettura gerarchica}

L'architettura gerarchica di Federated Learning rappresenta un'alternativa all'approccio centralizzato tradizionale, introducendo nodi di coordinamento intermedi, spesso regionali, per gestire gruppi di dispositivi edge,come mostrato in Figura~\ref{fig:arch-types}b. In questa configurazione, i dispositivi locali comunicano con i nodi regionali, che a loro volta gestiscono l'aggregazione e l'aggiornamento dei modelli per i dispositivi all'interno della propria area geografica o logica, prima di inviare le informazioni al server centrale. Questo approccio offre diversi vantaggi rispetto all'architettura centralizzata:
\begin{itemize}
\item \textit{Riduzione dei colli di bottiglia comunicativi}. I nodi regionali distribuiscono il carico di comunicazione, riducendo la congestione e i ritardi nelle trasmissioni di dati e aggiornamenti. Ciò rende l'architettura più scalabile, soprattutto per applicazioni di medie dimensioni, dove il numero di dispositivi è elevato ma non al livello delle grandi reti globali.
\item \textit{Efficienza energetica e di banda}. Con l’elaborazione regionale, i dispositivi edge non devono sempre comunicare direttamente con il server centrale, diminuendo il consumo di banda e di energia, elementi cruciali nei contesti IoT e mobili.
\end{itemize}

Tuttavia, l'architettura gerarchica conserva ancora una parziale centralizzazione, rappresentata dal server root della gerarchia, che rimane un singolo punto di fallimento. Se il server centrale subisce interruzioni, l’intero sistema potrebbe risentirne. Inoltre, questa centralizzazione, anche se parziale, può sollevare preoccupazioni di sicurezza e privacy, poiché la gestione delle informazioni aggregate è ancora concentrata in un nodo centrale.

\paragraph{Architettura regionale}
L'architettura regionale rappresenta un ulteriore passo verso la decentralizzazione nel Federated Learning, eliminando completamente il nodo di aggregazione centrale e affidando a nodi di aggregazione regionali il compito di gestire l'aggregazione dei modelli e la comunicazione tra i gruppi di dispositivi, Figura~\ref{fig:arch-types}c. Questo design presenta diversi vantaggi, tra cui:
\begin{itemize}
\item \textit{Miglioramento della robustezza}. L'assenza di un nodo centrale elimina il rischio di un singolo punto di guasto, aumentando la resilienza complessiva del sistema. In caso di interruzione di un nodo regionale, l’impatto è limitato al gruppo di dispositivi associati a quel nodo, senza compromettere l’intera rete.
\item \textit{Addestramento del modello più localizzato}. In applicazioni dove le distribuzioni dei dati sono più simili tra i dispositivi vicini (ad esempio, in contesti geografici o di settore specifico), i nodi regionali possono aggregare modelli che rispecchiano meglio le caratteristiche locali. Questo porta a modelli più accurati per specifici gruppi di dispositivi, migliorando le performance senza dover adattare eccessivamente il modello globale.
\end{itemize}

Questa architettura si adatta particolarmente bene a scenari con reti geograficamente distribuite o in cui le esigenze di privacy e sicurezza richiedono una gestione più autonoma dei dati tra i gruppi.

\paragraph{Architettura decentralizzata}
L'architettura decentralizzata nel Federated Learning sposta le responsabilità di aggregazione direttamente sui dispositivi edge, eliminando la necessità di nodi di coordinamento centrali o regionali, come mostrato in Figura~\ref{fig:arch-types}b. In questa configurazione, ogni dispositivo contribuisce autonomamente all'aggiornamento e alla condivisione dei modelli, distribuendo il carico computazionale e di comunicazione su tutti i nodi partecipanti. Questa struttura offre diversi vantaggi chiave, tra cui:
\begin{itemize}
\item \textit{Autonomia}. Poiché ogni dispositivo gestisce il proprio processo di addestramento e aggregazione, l'architettura è altamente autonoma. I dispositivi possono adattarsi rapidamente a cambiamenti locali, modificando il modello per rispondere a nuove condizioni ambientali o dati senza dipendere da un nodo centrale.
\item \textit{Adattabilità}. Con l’aggregazione distribuita, l'architettura è altamente flessibile e adatta a sistemi dinamici e scalabili come le reti IoT, dove i dispositivi possono connettersi e disconnettersi frequentemente. La configurazione decentralizzata consente una maggiore resilienza e un ridotto impatto dei singoli guasti sui risultati complessivi del sistema.
\end{itemize}

Tuttavia, questa architettura richiede un'infrastruttura robusta e presenta costi elevati. Ogni dispositivo deve avere la capacità di eseguire l'addestramento locale e supportare la comunicazione per la trasmissione dei modelli, caratteristiche che possono essere onerose in termini di risorse computazionali, energetiche e di rete. Nonostante questi requisiti, l'architettura decentralizzata rappresenta una soluzione efficace per applicazioni su larga scala e scenari in cui la privacy e la resilienza sono prioritarie, offrendo un sistema di Federated Learning scalabile e meno vulnerabile ai colli di bottiglia e ai punti di guasto singoli.
Nell'apprendimento interamente decentralizzato, l'interazione con un server centrale è completamente sostituita da una comunicazione \textit{peer-to-peer} tra i Client. In questa architettura, la topologia di comunicazione è rappresentata da un grafo connesso: i nodi del grafo sono i Client, mentre un arco tra due nodi indica la presenza di un canale di comunicazione diretto tra due Client. Il grafo della rete viene progettato per essere sparso e con un piccolo grado massimo, limitando il numero di connessioni per ciascun nodo. Questo design riduce il carico di comunicazione per ogni dispositivo, poiché ogni Client invia e riceve messaggi solo da un numero ristretto di peer vicini, riducendo l'onere computazionale e di rete.
In questo contesto, un round di apprendimento decentralizzato corrisponde a un ciclo in cui ciascun client: a) esegue un aggiornamento locale del modello usando i propri dati, e b) scambia il modello aggiornato o i parametri con i client vicini nel grafo.

Una caratteristica fondamentale dell'apprendimento decentralizzato è l'assenza di uno stato globale del modello, tipico delle architetture federate classiche. Qui, non esiste un unico modello centrale verso cui tutti i client convergono; al contrario, ogni client mantiene una versione locale del modello, che evolve progressivamente grazie agli scambi di parametri tra vicini. Questa decentralizzazione consente una maggiore flessibilità e robustezza, soprattutto in contesti con elevata dinamicità e dove i dispositivi si connettono e disconnettono frequentemente.\\

\begin{table}[t]
\centering
\begin{tabular}{|l|c|c|c|c|}
\hline
\textbf{Caratteristica} & \textbf{C} & \textbf{G} & \textbf{R} & \textbf{D} \\
\hline
Livello di centralizzazione & Alto & Moderato & Basso & Nessuno \\
\hline
Scalabilità & Limitata & Media & Alta & Molto alta \\
\hline
Punto singolo di guasto & Sì & Parzialmente & Ridotto & No \\
\hline
Collo di bottiglia comunicativo & Sì & Parzialmente & Ridotto & Minimo \\
\hline
Robustezza & Bassa & Media & Alta & Molto alta \\
\hline
Requisiti infrastrutturali & Bassi & Moderati & Moderati & Alti \\
\hline
Complessità di gestione & Bassa & Media & Alta & Molto alta \\
\hline
Dimensioni del sistema adatte & Piccole & Medie & Grandi & Molto grandi \\
\hline
\end{tabular}
\caption{Confronto sintetico tra le diverse architetture di Federated Learning. C: Architettura Centralizzata; G: Architettura Gerarchica; R: Architettura Regionale; D: Architettura Decentralizzata. \label{tab:architectures}}
\end{table}

\noindent Alla luce di quanto descritto per le architetture di riferimento nel Federated Learning, la Tabella~\ref{tab:architectures} sintetizza le caratteristiche peculiari di ognuna di esse.


\section{Apprendimento Federato Asincrono}\label{sub:sync-async}
L'approccio asincrono nel Federated Learning (AFL) rappresenta un’innovazione importante per superare le difficoltà di sincronizzazione e migliorare le prestazioni nei sistemi distribuiti. A differenza dell’approccio sincrono, dove l'aggregatore deve attendere che tutti i dispositivi completino l'addestramento locale prima di poter generare una nuova versione del modello e avviare il round successivo, nell'AFL il modello viene aggiornato ogni volta che un dispositivo invia i suoi parametri aggiornati. Questo meccanismo offre alcuni vantaggi~\cite{DBLP:journals/corr/abs-1903-03934}:
\begin{itemize}
\item \textit{Eliminazione degli strugglers}. Con l’AFL, i dispositivi più lenti (strugglers) non rallentano l'intero sistema, poiché il modello può essere aggiornato senza aspettare che tutti i client completino il loro addestramento. Questo migliora l'efficienza complessiva e consente un utilizzo ottimale delle risorse computazionali.
\item \textit{Evoluzione continua del modello}. Il modello si aggiorna in modo incrementale man mano che riceve nuove informazioni, consentendo una rapida progressione dell’apprendimento e una maggiore adattabilità del modello ai dati in arrivo.
\item \textit{Scalablità migliorata}. Ogni dispositivo può addestrare il modello secondo le proprie capacità computazionali e velocità, senza essere penalizzato dai tempi di elaborazione degli altri partecipanti. Questo rende AFL particolarmente adatto a sistemi su larga scala, dove i dispositivi hanno capacità eterogenee e partecipano in modo dinamico
\end{itemize}


\begin{figure}[h]
\centering
\includegraphics[width=1.0\textwidth]{sync_async_arch.png}
\caption{Confronto schematico tra aggregazione sincrona ed asincrona} \label{fig:sync-async}
\end{figure}

Nonostante i vantaggi in termini di gestione dell'etoregeneità dei dispositivi, l'approccio asincrono nel Federated Learning introduce nuove sfide. 
In primo luogo, la gestione della \textit{staleness} (osolescenza), ovvero il rischio che gli aggiornamenti inviati dai dispositivi possano risultare non allineati temporalmente con lo stato attuale del modello globale. Questo può comportare un aumento dell'errore durante la fase di aggregazione. Infatti, se un dispositivo invia un aggiornamento basato su dati o su un modello molto obsoleto rispetto all’ultima versione del modello aggregato, l'integrazione di tale aggiornamento può causare aumenti dell'errore complessivo del sistema. 
La mancata sincronizzazione potrebbe portare a situazioni in cui il modello globale converge verso punti stazionari non ottimali \cite{DBLP:journals/corr/abs-2007-07481}, compromettendo la qualità del modello finale, o peggio ad un instabilità degli aggiornamenti causando la divergenza del modello.
Inoltre, poiché i dispositivi non attendono gli altri per completare l'addestramento, si rischia di accumulare modelli che non rappresentano accuratamente il set di dati globale, specialmente in scenari di dati non-i.i.d. 

La staleness rappresenta un'importante sfida legata alla tempistica degli aggiornamenti dei modelli. Quando un dispositivo locale addestra un modello sui propri dati, utilizza una versione del modello globale che è stata inviata dal server centrale in un determinato momento. Tuttavia, a causa di vari fattori come differenze nelle capacità di calcolo, nella disponibilità della rete e nei tempi di completamento dell'addestramento, ci possono essere dei ritardi nell'invio degli aggiornamenti da parte dei diversi dispositivi.

Quando un dispositivo invia il proprio aggiornamento al server, questo aggiornamento può essere basato su un modello che non è più l'ultimo modello globale disponibile. Questo accade perché, nel tempo trascorso dal momento in cui il dispositivo ha iniziato l'addestramento locale, il server potrebbe aver ricevuto e aggregato aggiornamenti da altri dispositivi, aggiornando quindi il modello globale. In altre parole, l'aggiornamento che un dispositivo invia può riferirsi a uno stato del modello che è stato superato da modifiche più recenti apportate da altri dispositivi.

Una strategia comunemente utilizzata è quella di non aggregare aggiornamenti inviati da client strugglers. Come esaminato in \cite{DBLP:journals/corr/abs-1812-06127} per l'algoritmo FedAvg si nota che questa strategia, quando il numero di strugglers è elevato, porta ad una rallentata convergenza ed una possibile instabilità del modello.

L'effetto negativo degli strugglers è strettamente legato all'etereogeneità statistica. Nel caso in cui i dati siano i.i.d. la convergenza è più lenta ma comunque garantita. Come vedremo nella sezione relativa alle sperimentazioni effettuate, la distribuzione delle label dei dati sui vari Client non impatta in maniera significativa la convergenza in presenza di strugglers.

Per mitigare le sfide introdotte dall'approccio asincrono esistono tecniche ibride come FedBuff ed Hysync che vengono discusse nella sezione successiva.

\section{Ottimizzazioni note}
Per affrontare le problematiche del Federated Learning, sono stati sviluppati diversi algoritmi e strategie di ottimizzazione che mirano a migliorare la stabilità, l'efficienza e la robustezza del processo di apprendimento federato. 

\paragraph{FedProx - riduzione dell'eterogeneità statistica e di sistema}.
FedProx \cite{DBLP:journals/corr/abs-1812-06127} è un'estensione dell'algoritmo FedAvg progettato specificamente per affrontare l'eterogeneità dei sistemi e dei dati. FedProx introduce un termine di regolarizzazione $\mu$ che penalizza la differenza tra il modello locale di un Client e il modello globale. Questo termine di prossimità aiuta a mantenere gli aggiornamenti locali vicini al modello globale, anche quando i dati locali sono non-i.i.d., migliorando così la stabilità della convergenza. Inoltre, FedProx permette ai client di eseguire un numero variabile di iterazioni locali in base alle loro capacità computazionali, consentendo ai dispositivi meno potenti di partecipare senza l'obbligo di completare un numero fisso di epoche.

\paragraph{FedAvgM - miglioramento della convergenza con Momentum}
FedAvgM \cite{DBLP:journals/corr/abs-1909-06335}. è un'estensione dell'algoritmo FedAvg che introduce il concetto di Momentum lato Server per migliorare la stabilità e l'efficienza della convergenza, in particolare nei casi di eterogeneità statistica. Il Momentum è una tecnica comunemente utilizzata negli algoritmi di ottimizzazione per accelerare la discesa del gradiente, memorizzando una frazione della direzione di aggiornamento precedente e sommando questo termine al gradiente corrente. Utilizzando il Momentum, FedAvgM è in grado di attenuare le oscillazioni tra gli aggiornamenti dei Client, portando a una convergenza più rapida e stabile. Questa tecnica è particolarmente utile quando i dati sono non-i.i.d., poiché il Momentum aiuta a smussare le differenze tra gli aggiornamenti locali e a garantire una progressione più uniforme verso un modello globale ottimale.

\paragraph{FedBuff - miglioramento delle aggregazioni asincrone con buffer}
FedBuff \cite{DBLP:journals/corr/abs-2106-06639} è stato sviluppato per migliorare l'efficienza delle aggregazioni asincrone, mitigando l'impatto degli aggiornamenti obsoleti (staleness). FedBuff utilizza un buffer per accumulare e aggregare aggiornamenti dai 
Client in modo più organizzato e strategico, così che gli aggiornamenti vengano aggregati in momenti ottimali, riducendo al minimo l'effetto negativo della staleness sugli aggiornamenti del modello globale. Questo approccio migliora la stabilità del modello anche quando i dispositivi partecipano in modo asincrono, ottimizzando le risorse computazionali.


\paragraph{Hysync - gestione dell'eterogeneità di sistema con sincronizzazione ibrida}\label{fedopt:hysync}
Hysync \cite{9407951} è una soluzione ibrida che combina gli approcci sincrono e asincrono per affrontare l'eterogeneità di sistema e migliorare l'efficienza delle aggregazioni. Hysync permette di eseguire la sincronizzazione degli aggiornamenti in modo più flessibile, adattando il processo di aggregazione alle capacità dei client. Questo approccio riduce l'impatto dei client più lenti (strugglers) e garantisce che il modello globale continui a evolvere senza significative attese, mantenendo alta l'efficienza del processo di apprendimento.

\paragraph{FedNova - riduzione della variabilità con normalizzazione degli aggiornamenti}.
FedNova \cite{DBLP:journals/corr/abs-2007-07481} è stato proposto per affrontare i problemi di convergenza dovuti alla variabilità nelle iterazioni locali. FedNova introduce una tecnica di normalizzazione degli aggiornamenti dei client per assicurare che ogni contributo al modello globale sia equo e proporzionato al lavoro svolto, indipendentemente dal numero di epoche effettuate. Questo approccio è particolarmente utile per mitigare gli effetti negativi dell'eterogeneità di sistema e statistica, garantendo che ogni dispositivo contribuisca in modo equilibrato al modello globale, migliorando così la convergenza.

\paragraph{SCAFFOLD - riduzione della varianza per l'eterogeneità statistica}.
SCAFFOLD \cite{DBLP:journals/corr/abs-1910-06378} è un approccio progettato per ridurre la varianza introdotta dall'eterogeneità statistica dei dati, introducendo una strategia di correzione dei gradienti. SCAFFOLD utilizza una tecnica basata su una stima della differenza tra il gradiente locale e quello globale per stabilizzare gli aggiornamenti, riducendo significativamente la varianza e migliorando la stabilità della convergenza. Questo approccio consente una convergenza più rapida e precisa, rendendolo particolarmente efficace nei contesti in cui i dati locali differiscono notevolmente tra i dispositivi.

\paragraph{FedGroup - aggregazione per gruppi omogenei per una migliore personalizzazione}.
FedGroup \cite{9644782} propone un approccio innovativo per migliorare la personalizzazione e l’efficacia del Federated Learning in contesti caratterizzati da alta eterogeneità dei dati e dei dispositivi. Invece di applicare lo stesso modello globale a tutti i client, FedGroup suddivide i dispositivi in gruppi omogenei basati su caratteristiche comuni. Ogni gruppo elabora un modello intermedio più rappresentativo per i propri dati, che viene poi raffinato con aggiornamenti locali. Questo metodo riduce l’impatto della variabilità non-i.i.d. tra i Client, promuovendo una personalizzazione più accurata e una migliore convergenza del modello globale, preservando al contempo la privacy.
\clearpage{}
\clearpage{}\chapter{Stato dell'Arte}\label{chap:sota}

Negli ultimi anni, il Machine Learning e l'Intelligenza Artificiale hanno conosciuto un rapido sviluppo, portando a una crescente adozione di queste tecnologie in numerosi ambiti applicativi. Parallelamente, la necessità di preservare la privacy degli utenti e gestire efficacemente dati altamente distribuiti ha focalizzato l'interesse verso il Federated Learning (FL).

Per supportare questo approccio innovativo, sono stati sviluppati numerosi framework e librerie, ciascuno progettato per affrontare esigenze specifiche. Questi strumenti offrono funzionalità avanzate per gestire scenari di Cross-Silo e Cross-Device, due categorie principali che riflettono le diverse configurazioni del Federated Learning.

Come discusso nel Capitolo~\ref{chap:background:architetture}, Cross-Silo si  applica a scenari in cui i dispositivi partecipanti sono entità organizzative o server controllati, con una disponibilità stabile e risorse computazionali elevate tipiche in ambiti come la collaborazione tra istituti o aziende; Cross-Device si adatta a scenari in cui una vasta rete di dispositivi personali partecipa all'addestramento tipici di ambienti altamente eterogenei, caratterizzati da connessioni instabili, risorse limitate e partecipazione dinamica.

La selezione del framework più idoneo dipende strettamente dal tipo di scenario e dalle specifiche esigenze applicative. Alcuni framework si concentrano sull'ottimizzazione per scenari Cross-Silo, offrendo strumenti per il coordinamento e la gestione centralizzata dei dati, mentre altri sono progettati per affrontare la complessità e la scalabilità richieste in ambienti Cross-Device, integrando meccanismi per l'addestramento asincrono, la gestione dell'eterogeneità dei dispositivi e la robustezza agli errori.

\section{Framework consolidati}





Con lo sviluppo del Federated Learning, sono stati introdotti numerosi framework sul mercato, tra cui TensorFlow Federated (TFF)~\cite{tff}, Flower~\cite{flower}, FederatedScope~\cite{federated-scope}, PySyft~\cite{pysyft}, FedML~\cite{fedml} e OpenFL~\cite{}. Tuttavia, la maggior parte di questi framework si concentra principalmente sulla qualità del modello appreso, piuttosto che sulle ottimizzazioni legate alle prestazioni dell'infrastruttura distribuita e delle comunicazioni necessarie al processo federato. Molti di questi strumenti sono progettati per scenari Cross-Silo, mentre solo alcuni, come FederatedScope e Flower, sono in grado di affrontare le complessità di sistemi Cross-Device, caratterizzati da dispositivi non affidabili e operanti su larga scala.

Una proprietà in comune tra questi framework è il pattern architetturale master slave. Seguendo le linee del primo algoritmo di FL FedAvg \cite{DBLP:journals/corr/McMahanMRA16} questi framework si sono strutturati per seguire il suo approccio sincrono e coordinato dove esiste un entità centrale/decentralizzata che controlla e coordina i dispositivi federati.
Un framework che cerca di discostarsi da questa rigidità architetturale è FederatedScope il quale è implementato con un'architettura basata su eventi, ampiamente utilizzata nei sistemi distribuiti. Con questa architettura, un paradigma di FL può essere definito come coppie di eventi ed azioni: i partecipanti attendono determinati eventi (ad esempio, i parametri del modello vengono trasmessi ai client) per attivare i gestori corrispondenti (ad esempio, l'addestramento dei modelli basato sui dati locali). Pertanto, gli utenti possono esprimere i comportamenti di server e client dalla proprira prospettiva in modo indipendente, piuttosto che in modo sequenziale da una prospettiva globale.
Questa flessibilità permette di creare un sistema in grado di supportare scenari con un grande quantitativo di client e di definire diversi paradigmi di FL come l'apprendimento asincrono in maniera triviale.

Nel seguito descriviamo brevemente i principali framework di federated learning ad oggi disponibili.

\paragraph{TensorFlow Federated (TFF)} \cite{tff}. Sviluppato da Google, è uno dei framework più consolidati per il FL, costruito sull'ecosistema TensorFlow \cite{tf}. Offre building-block per l'implementazione di modelli, computazioni federate e gestione dei dataset in scenari Cross-Silo grazie alla sua capacità di gestire dati distribuiti in modo sicuro. 
TFF supporta la simulazione su più macchine e la creazione di architetture distribuite, ma il suo utilizzo reale in ambienti distribuiti su larga scala richiede ulteriori configurazioni e non è ottimizzato di default per tali scenari. TFF presenta alcune limitazioni. Ad esempio, non integra direttamente meccanismi di privacy avanzata come la \textit{Differential Privacy} (DF) o la \textit{Secure Multi-Party Computation} (SMPC), ma fornisce le basi per implementare tali tecniche. Gli sviluppatori devono configurare manualmente le strategie di privacy desiderate. Inoltre TFF non offre strumenti nativi per gestire attacchi come la manomissione dei dati (data poisoning) o la manipolazione degli aggiornamenti dei client (model poisoning). Pertanto, l'assenza di meccanismi predefiniti per mitigare questi attacchi rende vera questa parte dell'affermazione.

\paragraph{PySyft} \cite{pysyft}. Sviluppato da OpenMined, introduce funzionalità avanzate per la protezione della privacy, tra cui SMPC e DF. Questo framework, compatibile sia con PyTorch \cite{pytorch} che con TensorFlow \cite{tf}, consente la distribuzione su macchine singole o reti di nodi, utilizzando WebSocket per la comunicazione. PySyft è particolarmente utile per sviluppare applicazioni in cui la sicurezza dei dati è prioritaria.

\paragraph{SecureBoost} \cite{secure-boost}. E' un framework specializzato nella costruzione di alberi di decisione rinforzati in scenari con dataset partizionati verticalmente. Utilizzando strategie di cifratura, SecureBoost~\cite{secure-boost} consente a diverse parti di collaborare senza rivelare informazioni sensibili, garantendo un'elevata accuratezza in contesti distribuiti.


\paragraph{FederatedScope e FedML} \cite{federated-scope} \cite{fedml}. Offrono ulteriori livelli di flessibilità. FederatedScope, grazie a un'architettura basata su eventi, supporta strategie sincrone e asincrone, simulazione di attacchi e protezione della privacy. FedML, invece, integra protocolli di comunicazione come gRPC \cite{grpc} unito a Protocol Buffers \cite{proto-buff} (brevemente gRPC/Proto), MPI \cite{mpi} e MQTT \cite{mqtt}, adattandosi a diverse configurazioni, da dispositivi IoT a infrastrutture Cross-Silo ad alte prestazioni. Con i suoi moduli FedML-core e FedML-API, questo framework permette la simulazione autonoma, il calcolo distribuito e l'addestramento su dispositivo. 

\paragraph{LEAF} \cite{leaf}. Si distingue come strumento di benchmarking per il FL, fornendo dataset distribuiti e meccanismi di partizionamento utili per valutare le performance di altri framework.

\paragraph{OpenFL} \cite{openfl}. OpenFL è un framework open-source progettato per il Federated Learning (FL), che supporta sia il training distribuito che la protezione della privacy dei dati. Si distingue per la sua architettura modulare, che consente agli utenti di personalizzare facilmente i componenti del sistema e di integrare diversi algoritmi di apprendimento federato. OpenFL \cite{openfl} è compatibile con framework di deep learning popolari come TensorFlow \cite{tf} e PyTorch \cite{pytorch}, e offre una gestione avanzata della comunicazione tra nodi, riducendo la latenza e ottimizzando l'efficienza del sistema distribuito. È particolarmente adatto per scenari in cui le performance devono essere bilanciate con la protezione dei dati sensibili, come nelle applicazioni in ambito sanitario o finanziario.

\paragraph{Flower} \cite{flower}.  Flower è un framework federato flessibile che supporta l'addestramento distribuito su larga scala. Progettato per essere altamente modulare, Flower consente agli utenti di implementare e testare facilmente diverse strategie di apprendimento federato, come la sincronizzazione globale e le tecniche di aggregazione personalizzate. Flower è compatibile con framework di machine learning come TensorFlow \cite{tf}, PyTorch \cite{pytorch} e scikit-learn, e offre una comunicazione efficiente tra client e server tramite gRPC \cite{grpc}. Il suo design permette l'integrazione con diverse infrastrutture hardware e software, rendendolo ideale per scenari con dispositivi edge, IoT, o applicazioni su cloud. Flower è noto per la sua capacità di bilanciare flessibilità, facilità d'uso e scalabilità. \\ \\

\noindent Tutti i framework FL riportati supportano una modalità di simulazione, che consente di sperimentare e fare debugging di un sistema federato localmente. Tuttavia, non è scontato che supportino anche una modalità distribuita orientata al mondo reale, come ad esempio TFF, SecureBoost e OpenFL. Da questa prospettiva, il framework FL più limitato è LEAF: questo software è progettato esplicitamente per essere utilizzato solo per scopi di benchmarking.



\begin{table}[t]
    \centering
    \begin{tabular}{ |c|c|c|c|c| } 
    \hline
    \textbf{Framework} & \textbf{Cross-} & \textbf{Scenario} & \textbf{Protocol} & \textbf{Implem.} \\ 
    \hline
    TFF & silo & sim., real & gRPC/proto & Python \\
    PySyft & silo/device & sim., real & Websockets & Python \\
    SecureBoost & silo & simu., real & gRPC/proto & Python \\
    FederatedScope & silo/device & sim., real & gRPC/proto & Python \\
    LEAF & silo & sim. & - & Python \\
    FedML & silo & sim., real & gRPC/proto, MPI, MQTT & Python/C++ \\
    OpenFL & silo & sim., real & gRPC/proto & Python \\
    Flower & silo/device & sim., real & gRPC/proto & Python \\
    \hline
    \end{tabular}
    \caption{tabella riepilogativa di confronto tra i framework discussi \label{tab:frameworks}}
\end{table}


\subsubsection*{Comunicazione}


Un elemento centrale nel design di framework di Federated Learning è come viene implementata la comunicazione tra le differenti entità. La maggior parte degli strumenti, tra cui TFF, FedML, Flower e FederatedScope, utilizza protocolli come gRPC/Proto per la comunicazione tra Client e Server. Questo protocollo garantisce buone prestazioni, riducendo la latenza e gestendo in modo efficiente le richieste e risposte in scenari federati.

Framework come PySyft si differenziano adottando alternative come WebSocket, che fornisce connessioni bidirezionali persistenti. Questo approccio è particolarmente adatto per ottimizzare la comunicazione in tempo reale in scenari con requisiti di latenza ridotti.

FedML, invece, estende il supporto a protocolli aggiuntivi come MPI e MQTT, garantendo una maggiore adattabilità:
\begin{itemize}
\item \textbf{MPI} (Message Passing Interface) è ideale per ambienti ad alte prestazioni come i contesti Cross-Silo, in cui le reti e le risorse computazionali sono robuste (ad esempio, cluster con reti ad alte prestazioni).
\item \textbf{MQTT} (Message Queuing Telemetry Transport) si rivolge ai dispositivi IoT, offrendo comunicazioni a basso consumo energetico e a larghezza di banda ridotta, rendendolo adatto a scenari Cross-Device.
\end{itemize}

La scelta del framework e del protocollo di comunicazione dipende fortemente dalle caratteristiche dello scenario applicativo e dai vincoli di privacy, scalabilità e risorse disponibili che si vogliono ottenere. 
Nella Tabella \ref{tab:frameworks} sono riportate le caratteristiche principali di ciascun framework di Federated Learning, con l'obiettivo di evidenziarne le differenze e fornire un confronto diretto delle loro funzionalità, come il supporto per Cross-Silo o Cross-Device, lo scenario applicativo, i protocolli utilizzati ed il linguaggio di programmazione utilizzato per l'implementazione.







\subsubsection*{Apprendimento asincrono}



L'apprendimento federato asincrono rappresenta un'evoluzione significativa rispetto all'approccio sincrono tradizionale, offrendo vantaggi in termini di efficienza, scalabilità e gestione dell'eterogeneità tra i dispositivi partecipanti. In un ambiente asincrono, i dispositivi non devono attendere il completamento degli aggiornamenti di tutti i client prima di procedere, riducendo i ritardi e migliorando l'utilizzo delle risorse computazionali. Tuttavia, nonostante questi vantaggi, il supporto per l'apprendimento asincrono nei framework di Federated Learning è ancora piuttosto limitato.

La maggior parte dei framework attualmente disponibili, come TensorFlow Federated, FedML e OpenFL, sono stati progettati per operare in modalità sincrona e coordinata, con un forte focus sulla gestione dei round di comunicazione centralizzati. Questa struttura impone una sincronizzazione  tra i dispositivi, che risulta incompatibile con i requisiti dell'apprendimento asincrono.

Per supportare algoritmi asincroni, è necessaria una revisione significativa delle architetture sottostanti, in particolare dei meccanismi di aggregazione e comunicazione. Anche framework flessibili come Flower, sebbene offrano alcune funzionalità per scenari asincroni, richiedono modifiche sostanziali per un pieno supporto di questa modalità.

Attualmente, pochi framework supportano esplicitamente l'apprendimento asincrono:
\begin{itemize}
\item FedML: Include moduli sperimentali per il supporto di scenari asincroni, ma l'integrazione è limitata a contesti specifici e non è ottimizzata per grandi reti distribuite.
\item Flower: Consente una configurazione flessibile delle comunicazioni, il che facilita una certa implementazione di algoritmi asincroni. Tuttavia, manca un'infrastruttura nativa per la gestione di aggiornamenti asincroni complessi.
\end{itemize}

La ricerca in questo campo sta esplorando nuove architetture e protocolli di comunicazione per migliorare il supporto all'asincronismo, affrontando sfide come la \textit{staleness} degli aggiornamenti, l'aggregazione dinamica e la consistenza del modello globale. Con lo sviluppo di questi approcci, si prevede un'espansione del supporto per l'apprendimento asincrono nei framework futuri.

\section{Considerazioni}
Il panorama attuale del Federated Learning offre una vasta gamma di strumenti progettati per soddisfare esigenze applicative diverse. Framework come TensorFlow Federated, PySyft, FATE, Flower, OpenFL e FedML si distinguono per i loro differenti livelli di astrazione e facilità d'uso. Alcuni sono ottimizzati per implementazioni rapide e pronte all'uso, mentre altri si concentrano sulla personalizzazione e flessibilità, rendendoli adatti a utenti con competenze avanzate. Tuttavia, la gestione dell'asincronia nei contesti Cross-Device, caratterizzati da dispositivi eterogenei e connessioni instabili, rimane una sfida tecnica significativa.

In questo lavoro di tesi abbiamo voluto semplificare l'accesso a queste tecniche, offrendo soluzioni che possano essere facilmente implementate anche da un pubblico più ampio. Questo approccio non solo facilita la creazione di sistemi distribuiti ed eterogenei, ma promuove anche l'adozione del FL in scenari dove la privacy e la scalabilità sono requisiti importanti. In questo modo, si contribuisce a rendere il Federated Learning più accessibile riducendo le barriere tecniche per il suo utilizzo.


%
\clearpage{}
\clearpage{}\chapter{Architettura software del framework proposto}\label{chap:architettura}
Questo capitolo presenta l'architettura software del framework sviluppato in questa tesi, concepito per abilitare scenari di Federated Learning (FL) in ambienti altamente eterogenei e distribuiti. Il framework si distingue per il suo approccio modulare e il livello di astrazione che rende il suo utilizzo estremamente intuitivo per l'utente finale. L'obiettivo principale è quello di offrire un sistema che, pur gestendo la complessità intrinseca dell'apprendimento federato, mantenga un'interfaccia d'uso semplice e familiare, paragonabile a quella dei classici scenari di apprendimento centralizzato ed in alcuni casi ancora più intuitiva.

\subsubsection*{Scopi e Obiettivi}
Il framework proposto si prefigge di:
\begin{itemize}
    \item Semplificare l'adozione del Federated Learning: consentendo agli utenti di concentrarsi solo sugli aspetti fondamentali, come il flusso di dati di training, il modello da allenare ed una configurazione minimale.
    \item Gestire ambienti eterogenei: adattandosi a dispositivi con risorse computazionali e di rete variabili, sfruttando un approccio "resource-driven".
    
    \item Massimizzare la modularità e la flessibilità: offrendo un'architettura facilmente estendibile per supportare futuri miglioramenti, come approcci decentralizzati o algoritmi personalizzati.

    \item Ottimizzare le prestazioni: il framework è scritto in C, un linguaggio ad alte prestazioni, e adotta pattern di parallelizzazione e I/O non bloccante/asincrono, garantendo efficienza nella gestione delle comunicazioni e nell'elaborazione dei modelli, anche in scenari con numerosi client e grandi volumi di dati.

    \item Migliorare l’efficienza della comunicazione: attraverso l’uso del protocollo TCP ottimizzato, messaggi in formato personalizzato e tecniche avanzate come quantizzazione e compressione dei modelli, il framework riduce significativamente la latenza e il consumo di banda. Questi meccanismi permettono di supportare trasferimenti rapidi e scalabili anche in presenza di grandi quantità di client e aggiornamenti frequenti.
\end{itemize}

\begin{figure}[h]
\centering
\includegraphics[width=1.0\textwidth]{assets/astrazione1.png}
\caption{Schema astratto dal punto di vista dello sviluppatore}\label{fig:archAstr}
\end{figure}

Grazie al suo design, il framework consente agli sviluppatori di sfruttare le potenzialità del Federated Learning senza la necessità di addentrarsi nella complessità tecnica dell'implementazione sottostante. Questa filosofia "plug-and-play" è resa possibile da un alto livello di astrazione. L'idea originale è quella di fornire una soluzione "out of the box" in cui l'utente fornisce solamente 3 input (Figura~\ref{fig:archAstr}): Il flusso di dati per il training locale; Il modello da allenare; Ed una configurazione contenente informazioni di connessione, la strategia di aggregazione da utilizzare. Per non limitare la personalizzazione dei processi di apprendimento e quindi la flessibilità stessa di questo prodotto in realtà l'utente può fornire più configurazioni per esempio ridefinendo alcune funzioni logiche chiave del processo di apprendimento.

\newpage
\section{Resource driven learning}
Il framework proposto da questa tesi si discosta molto dall'approccio utilizzato da altre soluzioni disponibili discusse nella sezione stato dell'arte (cf. Capitolo~\ref{chap:sota}). Nel Federated Learning tradizionale è il server centrale che seleziona un sottoinsieme di dispositivi disponibili ed avvia e coordina il processo di apprendimento. Il processo federato viene organizzato in round, un round termina quando tutti i client (o una certa soglia, a seconda dell'implementazione) hanno inviato il proprio modello locale. Al termine di un round viene calcolato il nuovo modello globale e la procedura ricomincia. La nostra proposta sposta il controllo della partecipazione e dell'invio dei modelli locali ad i client stessi che li producono. Riteniamo che sia il client stesso l'entità che possa meglio valutare o meno quando sia il momento migliore per partecipare al miglioramento del modello globale. Per esempio quando si ritiene di aver raccolto una quantità di dati rilevante, oppure quando le risorse computazionali lo permettono (per esempio il livello di carica della batteria per un dispositivo mobile o IoT). Sono dunque le risorse dei dispositivi federati che guidano il processo di allenamento e non un'entità centrale: non è più l'aggregatore centrale che seleziona i client per partecipare all'allenamento del modello globale, ma i client che contribuiscono quando possono al suo miglioramento. 
Per permettere questo approccio "Resource driven" il sistema deve essere necessariamente asincrono, il che, come discusso in precedenza nella sezione background [\ref{fig:sync-async}], porta con sé vantaggi di scalabilità ma anche significativi problemi di instabilità in caso di eterogeneità statistica. Nella figura ~\ref{fig:flow-client} viene mostrato il processo di apprendimento dal punto di vista di un client federato, come si vede il processo di allenamento scaturisce da un evento locale, che sia temporale e non. L'evento "Force Sync" è un evento inviato dal server (questo può essere disabilitato in fase di configurazione) prima dell'aggregazione, come si vede nella figura ~\ref{fig:flow-server}, e server e serve per forzare l'arresto forzato della fase di training. Questo evento viene utilizzato per l'implementazione di algoritmi come HySync [\ref{fedopt:hysync}]

\begin{figure}[h] \label{fig:flow-client}
\centering
\includegraphics[width=15cm]{assets/flow-client.png}
\caption{UML Activity Diagram del processo di allenamento locale del modello federato}\label{fig:flow-client}
\end{figure}

\begin{figure}[h] \label{fig:flow-server}
\centering
\includegraphics[width=9cm]{assets/flow-server.png}
\caption{UML Activity Diagram del processo di aggregazione e creazione del modello globale lato server}\label{fig:flow-server}
\end{figure}

Il processo che invece segue il server è totalmente passivo e guidato dalla ricezione di aggiornamenti da parte dei client. 
Anche se fuori dallo scopo di questo framework, un classico approccio sincrono, come quello proposto dalla maggior parte delle soluzioni esistenti, potrebbe essere simulato. Per prima cosa il server deve essere configurato in modo che aggreghi i modelli quando un numero minimo di aggiornamenti è stato ricevuto. Questa è una configurazione nativamente supportata in quanto il server utilizza una strategia di bufferizzazione degli aggiornamenti. I client, come il server, non subiscono grandi cambiamenti: è sufficiente sostituire l'evento che da inizio all'allenamento del modello con la notifica della presenza di un nuovo modello globale inviata dal server. Qui sotto viene riportato uno pseudo-codice che mostra quanto appena detto.
\begin{lstlisting}[language=Python]
loop:
    wait_for_new_model()
    model = get_latest_global_model()
    local_model = train(model, data)
    send_update(local_model)
\end{lstlisting} 
\newpage
\section{Architettura Software Server}
L'obiettivo di questa tesi è progettare un framework capace di gestire scenari di Federated Learning in ambienti altamente eterogenei e distribuiti, che possano coinvolgere un elevato numero di client. Sebbene il nostro approccio iniziale impieghi un'architettura centralizzata, essa è progettata in modo modulare per supportare future implementazioni scalabili, affidabili e resilienti. L'architettura centralizzata viene adottata in questa fase come un punto di partenza per testare la fattibilità della soluzione proposta. In futuro, il framework potrà evolversi verso un'architettura federata più decentralizzata, con nodi più distribuiti e con sistemi di high availability. Vedi sezione sulle architetture~federate~\ref{chap:background:architetture}.

\paragraph{Linguaggio e Design}
La scelta del linguaggio di implementazione del server è cruciale per ottenere prestazioni elevate, specialmente in un contesto di Federated Learning dove la gestione delle comunicazioni e l'elaborazione dei modelli richiedono grande efficienza. A differenza di molte soluzioni esistenti che utilizzano Python (cf. Capitolo ~\ref{chap:sota}), il nostro framework è implementato in C, un linguaggio di sistema noto per le sue performance. L'uso di C ci permette di ridurre al minimo il tempo di esecuzione per le operazioni più critiche e consente una gestione più fine della memoria e delle risorse. L’architettura del software è progettata con un focus sull’adozione di pattern orientati alle performance, come il pattern Producer-Consumer, per gestire le code di aggiornamenti dei modelli e migliorare l’efficienza nelle operazioni di I/O e pattern Data Parallel per ridurre latenze e migliorare il throughput.
Nella Figura~\ref{fig:server-structure} è illustrata ad alto livello la struttura del sistema software lato server, le cui componenti vengono introdotte nei paragrafi successivi.


\begin{figure}[h]
\centering
\includegraphics[width=1.0\textwidth]{assets/server-structure.png}
\caption{Architettura software del Server.}\label{fig:server-structure}
\end{figure}


\subsubsection*{Comunicazione Client-Server} 
Per la gestione della comunicazione tra Server e Client, il sistema adotta il protocollo TCP, che garantisce una trasmissione affidabile dei messaggi. La scelta di TCP è motivata dalla necessità di avere un canale sicuro e stabile per l'invio di modelli e aggiornamenti tra Client e Server. Ogni messaggio di comunicazione tra il Server e i Client è codificato secondo un formato definito ad-hoc, per ottimizzare l'efficienza e ridurre la latenza.
Per ottimizzare la gestione di un numero elevato di client e richieste contemporanee, abbiamo adottato un'architettura basata su eventi e I/O non bloccante, sfruttando l'API \textit{epoll} \cite{epoll}, una funzione della libreria standard di Linux. Epoll è particolarmente vantaggiosa in scenari ad alta concorrenza, in quanto consente di monitorare un numero elevato di file descriptor senza necessità di operazioni di polling ripetute, migliorando significativamente le performance rispetto ad altri metodi come \texttt{select} e \texttt{poll}.
L'uso di epoll permette di gestire in modo efficiente le connessioni I/O, notificando il Server solo quando ci sono dati da leggere o scrivere, evitando il blocco del processo principale e migliorando la reattività del sistema. Inoltre, epoll è altamente scalabile, in quanto supporta la gestione di migliaia di connessioni simultanee con un overhead minimo.
Per incrementare ulteriormente la capacità del server di gestire un numero maggiore di Client, è possibile, come illustrato in Figura~\ref{fig:server-structure}, utilizzare più thread (tipicamente denominati \textit{Workers}) per distribuire il carico di lavoro. In questo contesto, il kernel Linux consente di accettare connessioni sulla stessa porta da più thread senza necessità di meccanismi di sincronizzazione complessi, grazie al parametro \texttt{SO\_REUSEADDR}. Questo parametro permette a diversi thread di legarsi alla stessa porta, rendendo possibile l'uso di un pool di thread per gestire in parallelo le connessioni in ingresso. Le connessioni vengono distribuite dal sistema operativo in modo Round Robin, assegnando a ciascuna connessione un thread specifico per l'elaborazione. In questo modo, ogni connessione è completamente isolata e gestita da un singolo thread, garantendo una gestione parallela e scalabile delle richieste.
Questa architettura rende il sistema estremamente efficiente su architetture multi-core ed  è capace di scalare orizzontalmente, consentendo di gestire centinaia di migliaia di connessioni simultanee in modo fluido e reattivo.

\subsubsection*{Aggregazione Asincrona dei Modelli}
L'aggregazione dei modelli avviene in modo asincrono. Quando arrivano aggiornamenti dai Client, questi vengono dapprima decompressi e normalizzati, per poi essere immagazzinati in un buffer. La strategia di buffering decide quando è il momento opportuno per procedere con l'aggregazione dei modelli, in base alla quantità di aggiornamenti ricevuti e alla logica di gestione del flusso. Una volta che il momento giusto per l'aggregazione è stato identificato, vengono assegnati i pesi ai modelli in base all'algoritmo di ottimizzazione federata utilizzato. Ad esempio in FedAvg il peso di ciascun modello è proporzionale al numero di campioni su cui è stato addestrato. I pesi possono dipendere da altri fattori legati alla specifica strategia di ottimizzazione. Dopo l'assegnazione dei pesi, si esegue una media ponderata dei modelli.

Sia la fase di aggregazione che quelle di pre-processing e averaging dei modelli possono essere eseguite in parallelo. La fase di pre-processing può adottare un pattern che combina i costrutti \textit{Farm} e \textit{Pipelining}, mentre la fase di averaging può essere gestita tramite il pattern \textit{Reduce}. Questi approcci consentono di ottimizzare l’elaborazione dei modelli, sfruttando al massimo la parallellizzazione per migliorare le performance complessive.

Per gestire grandi volumi di aggiornamenti e modelli di dimensioni elevate, tutte le operazioni relative al modello fanno uso della paginazione su disco utilizzando "Memory Mapped IO". Questo approccio consente al sistema di evitare il sovraccarico della memoria principale, permettendo una gestione più efficiente delle risorse, soprattutto quando si trattano dati di grandi dimensioni. Quando vengono aggregati i modelli per esempio questi sono salvati sul disco ed una pagina alla volta vengono aggregati e viene generato il nuovo modello, questo permette al server di poter gestire grandi quantità di aggiornamenti senza mai eccedere la quantità di memoria disponibile.

\subsubsection*{I/O Overhead e Gestione della Memoria}
La comunicazione tra i sottosistemi di comunicazione e aggregazione avviene tramite strutture dati in memoria condivisa. Ogni aggiornamento viene ricevuto suddiviso in chunk ordinati per ridurre la latenza nella gestione dei pacchetti e prevenire il sovraccarico della memoria. Ogni chunk dopo essere ricevuto viene quindi inserito in una coda di elaborazione.

Un thread specializzato gestisce il trasferimento degli aggiornamenti sul disco, utilizzando I/O asincrono per ridurre l'overhead derivante dal context switch tra il kernel e lo spazio utente, nonché dall'uso dell'I/O bloccante. Una volta che un aggiornamento è completamente scritto su disco (tutti i suoi chunk sono stati salvati), il suo identificatore viene inserito nella coda degli aggiornamenti, che viene successivamente letta dal sottosistema di aggregazione per essere processata.

Un approccio simile viene adottato per l'invio dei modelli ai client, al fine di ridurre l'overhead associato alla compressione e invio dei modelli. Quando una richiesta di modello viene ricevuta, essa viene inserita nella coda del gestore dei modelli, mentre il socket viene momentaneamente disabilitato. Il gestore dei modelli possiede una cache, che contiene modelli precedentemente salvati nel file system. Se il modello richiesto è già presente nella cache, viene immediatamente inviato al client. Se il modello non è disponibile, la richiesta viene aggiunta alla coda di compressione. Una volta completata la compressione, la richiesta di invio del modello viene riaggiunta alla coda del gestore dei modelli per l'invio effettivo.

La cache gioca un ruolo cruciale nelle performance del sistema. Ad esempio, quando il modello globale viene aggiornato, è molto probabile che si ricevano richieste per lo stesso modello da più client, e l'uso della cache permette di ridurre significativamente i tempi di risposta e l'overhead computazionale. In prospettiva futura, questo sottosistema potrebbe essere esternalizzato dal Server centrale e implementato attraverso una Content Delivery Network (CDN), migliorando ulteriormente la distribuzione e la scalabilità del sistema.

\subsubsection*{Eventi di sincronizzazione}
La gestione dell'invio eventi come ForceSync e NewGlobalModel viene gestitia attraverso un pattern publish-subscribe. Esiste un messaggio speciale di upgrade che rimuove la connessione dalla normale gestione di epoll e aggiunge il socket ad una lista di listeners. Cosí facendo viene ceduta l'ownership della connessione al thread di aggregazione.

\subsubsection*{Configurazione e Personalizzazione}
La configurazione e la personalizzazione del framework per adattarlo a qualsiasi scenario ad algoritmo di federated learning avviene tramite la definizione di due funzioni da parte dell'utilizzatore.

\lstset{style=longBlock}
\begin{lstlisting}[language=C]
typedef struct
{
    uint8_t type; // 0: NO, 1: YES, 2: WAIT
    time_t until;
} agg_config_t;

agg_config_t should_aggregate_models(updates_t updates, time_t *last_aggregation, func_t force_sync)

double get_update_weight(model_info_t local_model, model_t gobal_model)
\end{lstlisting}
\lstset{style=codeStyle}

La funzione should\_aggregate\_models definisce la strategia di bufferizzazione. L'utente in base agli aggiornamenti pronti per essere aggregati decide se aggregare subito i modelli, se non aggregare ed aspettare aggiornamenti successivi o se aspettare un lasso di tempo ed eseguire l'aggregazione se nessun aggiornamento viene ricevuto nel tempo specificato. Una funzione force\_sync viene passata per permettere all'utente di mandare una notifica ad i client che non hanno ancora terminato l'addestramento del modello. La funzione get\_update\_weight invece definisce la strategia di aggregazione, riceve in input le informazioni del modello locale ed il modello globale e ritorna il peso che il modello locale deve avere nel passaggio successivo di averaging. Il modello con se, oltre che ai pesi, contiene anche dei metadati che possono essere utilizzati per personalizzare maggiormente il processo di aggregazione. Per implementare un algoritmo di FedAvg per esempio è necessario un metadato che contenga la dimensione del dataset su cui il modello locale è stato allenato. Per garantire la correttezza dei metadati è necessario definire una funzione (is\_valid\_metadata) che può essere utilizzata per validare il contenuto. Essendo questa una tesi che mira a studiare la fattibilità della soluzione proposta per il momento queste funzioni devono essere definite all'interno del codice sorgente, ciò richiede la conoscenza del linguaggio C e la ricompilazione manuale del progetto. Per il futuro l'idea però è quella di distribuire l'applicazione pre compilata e di permettere la personalizzazione attraverso linguaggi di scripting più ad alto livello facendo affidamento a librerie come Extism \cite{exitism} che permettono all'utente di definire queste funzioni in qualsiasi linguaggio l'utente preferisca (Python, JS, C, C++, Java, Ruby, Go ... etc). Oltre che alla possibilità di scripting, l'idea è che il framework supporti già nativamente le strategie più comuni.

 
\newpage
\section{Protocollo di comunicazione}
Il protocollo di comunicazione client-server necessario per l'implementazione della soluzione proposta in questo elaborato è molto semplice e necessita di un numero limitato di messaggi. Per questa ragione abbiamo optato per un'implementazione semplice che non faccia utilizzo di tecnologie come gRPC, HTTP, Websocket, MPI etc. Per la comunicazione tra client e server abbiamo dunque utilizzato dei socket TCP con messaggi in formato non standard e ottimizzato per questo caso d'uso. Nella figura~\ref{fig:proto-message} viene mostrata la struttura di un messaggio generico.
\begin{figure}[h] 
\centering
\includegraphics[width=14cm]{assets/proto-message}
\caption{Struttura di un messaggio del protocollo di comunicazione}\label{fig:proto-message}
\end{figure}

Il protocollo è costituito da 4 messaggi principali: Autenticazione, Richiesta del modello globale, Invio di un aggiornamento, Sottoscrizione al canale di notifica.

\paragraph{Formato dei modelli}
Il framework proposto è progettato per essere completamente trasparente per l'utente, consentendogli di utilizzare liberamente qualsiasi backend di machine learning desideri, come PyTorch, TensorFlow o altri. Tuttavia, ciascuno dei principali framework di machine learning utilizza un formato proprietario per salvare lo stato dei modelli allenati, e questi formati non sono ottimizzati per contesti di apprendimento federato, in cui i pesi dei modelli vengono frequentemente trasmessi tra dispositivi e dove l'uso efficiente della banda di rete è cruciale. In un contesto di apprendimento federato, i pesi del modello devono essere condivisi frequentemente tra i nodi partecipanti e il server centrale. La continua trasmissione di modelli allenati rappresenta un carico significativo per la larghezza di banda della rete, il che rende necessaria una strategia che minimizzi il volume dei dati trasmessi. Inoltre, per supportare algoritmi di ottimizzazione federata avanzati, come Federated Averaging (FedAvg), è fondamentale arricchire il modello con metadati che descrivano il contesto di addestramento, ad esempio la dimensione del dataset utilizzato in ciascun nodo. Queste informazioni consentono di attribuire un peso appropriato agli aggiornamenti inviati da ciascun partecipante, migliorando così la precisione e l'efficacia dell’ottimizzazione.

Per soddisfare questi requisiti, abbiamo scelto di sviluppare un formato file personalizzato, ottimizzato sia per l'efficienza nell'uso della banda di rete sia per l'integrazione di metadati e informazioni strutturali, necessarie per una gestione efficace degli aggiornamenti dei modelli nel contesto dell’apprendimento federato. In Figura~\ref{fig:model-format} è illustarta la struttura del file. 

\begin{figure}[h] 
\centering
\includegraphics[width=9cm]{assets/model-format.png}
\caption{Struttura del file che rappresenta lo stato di una rete ML}\label{fig:model-format}
\end{figure}

Il formato del file è progettato con una struttura a due componenti principali: un'intestazione e un corpo. Questa organizzazione permette un accesso rapido alle informazioni chiave sul modello fin dal primo byte trasmesso, consentendo un'elaborazione efficiente anche durante il trasferimento (Questo permette di gestire gli aggiornamenti come una stream di dati). L’intestazione contiene tutte le informazioni necessarie per descrivere il modello, ed è strutturata in modo da non essere compressa. Questa scelta permette al server di accedere a tutte le informazioni essenziali non appena viene ricevuto il primo chunk di dati, consentendo una gestione ottimizzata della rete e un’interpretazione immediata del file. 
L'intestazione è composta da:
\begin{itemize}
    \item Informazioni sul file: versione del formato, dimensioni dei vari segmenti del file e flag che descrivono proprietà come la compressione, la quantizzazione e la presenza di sezioni opzionali.
    
    \item Metadati: una sezione dedicata a informazioni di vario tipo, come stringhe, interi o float, che descrivono il contesto di addestramento del modello. Questi metadati possono includere dettagli come il numero di campioni usati nel training locale o la configurazione specifica del nodo. Questi dati consentono di calibrare correttamente gli aggiornamenti dei modelli, particolarmente utile per algoritmi di aggregazione come FedAvg, che richiedono informazioni sulla dimensione del dataset.
    
    \item Descrizione della struttura dei tensori: una sezione opzionale che fornisce dettagli sulla struttura dei tensori, come dimensioni e layout. Questa sezione può essere disabilitata tramite un flag se il destinatario conosce già tali informazioni, riducendo così ulteriormente la quantità di dati trasmessi. Le descrizioni strutturali sono utili per garantire che le operazioni di aggregazione o fusione sui modelli siano effettuate con precisione, senza rischi di interpretazioni errate dei dati.
\end{itemize}

Il corpo del file contiene i valori dei tensori, ossia i pesi e i parametri del modello, in un formato sequenziale organizzato in ordine row-major. Questa disposizione permette di trattare i pesi come un unico grande vettore, facilitando le operazioni di aggregazione come la media tra i modelli: i parametri possono essere sommati direttamente, senza dover conoscere dettagli specifici sulla struttura interna dei tensori. Questa semplificazione rende il formato altamente flessibile e adatto ad ambienti federati, dove le strutture dei modelli tra i nodi devono essere allineate rapidamente. Inoltre questa struttura permette di inviare il modello in modalità streaming non richiedendo di tenere tutto il modello in memoria, questo aiuta specialmente lato server perché permette di aumentare il numero di client concorrenti che possono essere gestiti.

Il formato è progettato per facilitare sia la compressione che la quantizzazione dei dati, due tecniche essenziali per ridurre al minimo il consumo della banda di rete. L'intestazione include un campo chiamato diffed\_model\_version, che indica la versione globale del modello da cui deriva la versione corrente. Grazie a questa funzionalità, i pesi del modello possono essere salvati come differenza (o "diff") rispetto a una versione di riferimento, anziché come valori assoluti. Questo approccio riduce drasticamente la quantità di dati, poiché molte differenze risultano vicine a zero, rendendo il file facilmente comprimibile con algoritmi di compressione senza perdita.

Questa struttura di compressione incrementale, unita alla possibilità di quantizzare i pesi (ad esempio, utilizzando una precisione inferiore come int8 anziché float32), consente di ottimizzare ulteriormente la trasmissione del modello nel contesto dell’apprendimento federato, riducendo i requisiti di rete senza sacrificare l'accuratezza complessiva del modello.

 
\newpage
\begin{figure}[t]
\centering
\includegraphics[width=10cm]{assets/file-system.png}
\caption{Approccio di Client basato su file system}\label{fig:file-system}
\end{figure}

\section{Architettura Software del Client}
L'architettura software del client è estremamente semplice, consiste in una libreria che nasconde l'interazione con il server centrale.  La sua interfaccia è il più minimale possibile, come si vede nel codice sottostante, basta importare la libreria e definire un Client. 

\lstset{style=longBlock}
\begin{lstlisting}[language=Python]
token = "auth_token..."
store = BlobStore()
client = FLClient("coordinator.domain", store, token, backend="pythorch")
net = client.get_model()

client.train(rounds=N, dataloader)
\end{lstlisting}\label{code:client}
\lstset{style=codeStyle}

Dopodiché sono disponibile due funzioni: get\_model permette di scaricare il modello globale e di inizializzare la rete nel backend selezionato; train permette di eseguire automaticamente N round di apprendimento federato. La struttura del modello e l'implementazione della strategia di training sono definite globalmente e fornite dal server centrale. Queste funzioni sono personalizzabili in fase di configurazione del server ma l'idea è quella di fornire automaticamente le funzioni relative alle ottimizzazioni più comuni come FedAvg, FedProx, FedNova etc. È comunque possibile riscrivere completamente la funzione di training ed usare in modo esplicito le funzioni di ricezione eventi ed invio dei modelli.



L'idea per il futuro è quella di rendere ancora più trasparente ed agnostico il framework. Un modo per farlo potrebbe essere quello di creare una soluzione che non dipenda dalla tecnologia implementativa dell'applicazione. Per far questo si potrebbe creare un deamon che legga i dati da una cartella del file system ed in background esegua l'apprendimento federato. Inoltre lo stesso deamon avrebbe il compito di mantenere aggiornato il modello globale utilizzato. In Figura~\ref{fig:file-system} è rappresentata l'architettura ad alto livello della proposta. Con questa soluzione sviluppare applicazioni che facciano uso di federated learning sarebbe molto semplice, in quanto l'applicazione dove solo salvare i dati raccolti in una cartella. 


 \clearpage{}
\clearpage{}\chapter{Valutazione sperimentale}\label{chap:valutazione}

Prima di procedere con l'implementazione del framework descritto in questa tesi, abbiamo condotto una serie di esperimenti preliminari per valutare la fattibilità empirica del nostro approccio. Come approfondito nel Capitolo~\ref{chap:architettura}, il framework richiede l'adozione di un approccio asincrono per soddisfare i requisiti di un sistema caratterizzato da forte eterogeneità e da comunicazioni non pienamente affidabili. Tuttavia, come discusso nella sezione~\ref{sub:sync-async}, l'adozione di un approccio asincrono introduce numerose sfide legate alla convergenza e alla stabilità del modello, problematiche che possono essere ulteriormente aggravate in presenza di configurazioni eterogenee dei Client.

Per valutare l'impatto dell'asincronia sul sistema, abbiamo condotto esperimenti iniziali in contesti sincroni o ibridi, introducendo progressivamente assunzioni più rilassate. Questo approccio graduale ci ha permesso di comprendere in maniera più approfondita le criticità derivanti dall'eterogeneità, offrendo una base solida per affrontare le problematiche che emergono in ambienti di esecuzione asincroni. 

\section{Architettura dei test}
Per valutare l’approccio proposto, abbiamo progettato un’architettura di test su un cluster di 16 nodi omogenei per condurre esperimenti in un contesto distribuito e con numerosi Client. Questa scelta differisce significativamente da altri framework di simulazione dell’apprendimento federato, come Flower, che tipicamente eseguono le simulazioni su una singola macchina utilizzando processi o thread separati per rappresentare i Client. La nostra soluzione è progettata per replicare in modo realistico le condizioni operative di un sistema distribuito, offrendo quindi una maggiore fedeltà a scenari realistici e maggiore scalabilità orizzontale.

\subsubsection*{Architettura del Cluster} 
L’architettura distribuita è composta da:
\begin{itemize}
    \item Un nodo dedicato al Server Ticker (le cui funzionalità verranno introdotte nella prossima sezione) ed all’Aggregatore, per il coordinamento dei Client e la gestione asincrona degli aggiornamenti.

    \item Quindici nodi Client, ciascuno rappresentante un’entità indipendente del sistema federato, che esegue il ciclo di allenamento locale ed effettua la comunicazione con l’Aggregatore.
\end{itemize}

Ogni nodo dispone di una CPU a 8 core con hyperthreading e memoria sufficiente per eseguire i carichi di lavoro assegnati. La comunicazione tra i nodi avviene tramite una rete locale ad alta velocità di tipo Ethernet. L’architettura distribuita proposta offre una scalabilità nativa, rendendo il framework adatto per esperimenti di dimensioni crescenti senza richiedere modifiche strutturali. In particolare:

È possibile aggiungere nuovi nodi Client al cluster con un impatto minimo sull’ infrastruttura esistente.
L’utilizzo di nodi fisici permette di sfruttare pienamente le risorse hardware, migliorando le prestazioni e riducendo il rischio di generare colli di bottiglia dovuti all'esecuzione concorrente di numerosi Client su un solo server.
Grazie alla scelta di eseguire i test su un cluster distribuito, il nostro framework rappresenta un approccio più realistico e scalabile rispetto alle soluzioni basate su simulazioni centralizzate, dimostrando la sua capacità di operare efficacemente in ambienti complessi e su larga scala.

\subsubsection*{Componenti dell’Architettura}
Il sistema è suddiviso in tre componenti principali: il Server Ticker, i Client ed un Aggregatore (Sever del framework ampiamente descritto nel Capitolo~\ref{chap:architettura}) che gestisce l’apprendimento federato in modalità completamente asincrona.

Componenti principali:
\begin{itemize}
    \item Server Ticker: ha il compito di sincronizzare il comportamento dei Client, senza partecipare direttamente al processo di aggregazione. Le sue funzioni principali sono: 
    \begin{itemize}
        \item Invio di Tick: invia segnali periodici (tick) a tutti i Client connessi, indicando l’inizio di un nuovo ciclo di computazione.
        
        \item Sincronizzazione dei Client: attende gli ack dai Client, che confermano la ricezione del tick. Prima di inviare un nuovo tick, il Server Ticker aspetta che i Client abbiano completato il ciclo corrente o abbiano gestito eventuali ritardi.
    \end{itemize}

    Il Server Ticker non interagisce con il modello globale né raccoglie aggiornamenti: la sua funzione è esclusivamente quella di coordinare i Client scandendo il tempo.


    \item Client: Ogni nodo del cluster esegue una o più istanze di Client, che hanno il compito di: a) ricevere il modello globale dall’Aggregatore; b) eseguire il ciclo di allenamento locale; c) inviare gli aggiornamenti del modello all’Aggregatore.

    Il comportamento sincrono o asincrono dipende direttamente dai Client. Quest'ultimi possono:
    \begin{enumerate}
    \item inviare immediatamente un ack al tick ricevuto, consentendo al Server Ticker di proseguire con il prossimo ciclo (comportamento asincrono).
    \item ritardare l’invio dell’ack, introducendo uno stallo controllato nel sistema o una sua parte e quindi forzando la sincronizzazione di solo alcune fasi dell'apprendimento federato.
    \end{enumerate}

    \item Aggregatore: componente completemente asincrono che gestisce esclusivamente la ricezione degli aggiornamenti dai Client,  l'aggregazione e la distribuzione dei modelli globali. Non vi è alcun coordinamento diretto tra Aggregatore e Server Ticker dato che la sincronizzazione tra i Client è gestita esclusivamente dal Ticker.
\end{itemize}

L'architettura di test è mostrata ad alto livello della Figura~\ref{fig:test-arch}

\begin{figure}[h]
\centering
\includegraphics[width=1.0\textwidth]{assets/test-arch.png}
\caption{Architettura dei test}\label{fig:test-arch}
\end{figure} \section{Problemi di eterogeneità in contesti sincroni}
Nella maggior parte dei paper di ricerca si affronta il tema dell' eterogeneità dei sistemi di calcolo per quanto riguarda Federated Average (FedAvg) semplicemente ignorando gli aggiornamenti in ritardo. Per aggiornamenti in ritardo si intendono gli aggiornamenti che sono iniziati da una versione del modello globale che non è più l'ultima al momento del loro invio. I Client che inviano aggiornamenti in ritardo vengono detti stragglers. Essendo l'intento finale quello di sviluppare un sistema asincrono volevamo capire l'impatto sulla convergenza nel caso in cui si considerino comunque tali aggiornamenti.

Per fare ciò in questa sezione studiamo l'impatto della presenza di ritardi e dell' aggregazione di aggiornamenti obsoleti aggiungendo in modo incrementale problematiche di eterogeneità (sia statistica, che di sistema). Per confermare i risultati degli esperimenti ognuno di essi è stato ripetuto 8 volte; per praticità e compattezza di seguito verranno riportati i grafici di uno solo di essi.

\subsubsection*{Pattern di ritardo}
Per rendere gli esperimenti discussi successivamente deterministici e quindi replicabili, abbiamo definito 3 pattern principali di ritardo:
\begin{itemize}
    \item \textbf{Campionamento}: 
        Consideriamo $N_{t}$ la quantità di Client attivi al round $t$, $\lceil N_t * p \rceil$ è il numero di Client stragglers in quel round $t$.
        
    \item \textbf{Latenza}: 
        Questo pattern utilizza due parametri $L, p$: $L$ descrive la quantità di ritardo che un client straggler ha rispetto al modello globale. Per esempio $L=2$ implica che un Client straggler invia un aggiornamento relativo al modello globale vecchio ormai di due generazioni. $p$ descrive la probabilità che un Client sia uno straggler. Per esempio se ci sono $100$ Client e $p = 0.5$, 50 Client sono stragglers. Questo pattern ci aiuta a capire come varia la stabilità della convergenza del modello nel caso in cui ci siano tipologie diverse di Client che hanno tempi diversi di elaborazione durante l'intero apprendimento federato.
  

    \item \textbf{Random}: 
        Questo pattern fa uso di un generatore di numeri pseudo casuali. Ad ogni round ogni client ha una probabilità $p$ di essere in ritardo. Il numero di generazioni di ritardo è dunque dato dalla distribuzione geometrica di parametro $1-p$.

\end{itemize}


\subsubsection*{Stragglers con dati IID}
Per gettare le basi ed un riferimento di base per le sezioni successive abbiamo iniziato con il caso base: apprendimento sincrono su dati IID.
L'approccio utilizzato non è teoricamente totalmente sincrono ma lo è in pratica: anche se vengono permessi aggiornamenti relativi a vecchie versioni del modello globale. Infatti non viene prodotto un modello globale ogni qual volta un aggiornamento viene ricevuto ma, quando tutti gli aggiornamenti dei Client non stragglers vengono ricevuti.

Per i primi test abbiamo utilizzato i dataset MNIST (ML di riconoscimento di immagini) ed AGNews (ML di classificazione del testo). Abbiamo utilizzato AGNews per verificare l'indipendenza dei risultati dal tipo di ML e di dataset. Per compattezza d'ora in avanti verrà mostrato solamente il caso di MNIST (AGNews ha risultati che confermano lo stesso andamento). I dataset sono stati divisi uniformemente su $N$ Clients. Per uniformemente si intende che ogni client ha lo stesso numero di campioni per ogni classe. Per notare esclusivamente l'effetto degli stragglers ed isolare la variabile di eterogeneità di sistema abbiamo utilizzato gli stessi parametri di apprendimento per ogni client (Epochs=6, BatchSize=256, LearningRate=0.01). 

Essendo lo scopo della tesi quello di proporre un framework in grado di gestire un grande numero di client, abbiamo utilizzato $100$ Client, a differenza di studi condotto in altre ricerche come \cite{DBLP:journals/corr/abs-1812-06127} in cui viene utilizzato un numero limitato di Client (nel range di poche decine: $Client \in [10, 30]$). 

Abbiamo ripetuto l'esperimento variando i pattern di ritardo. In Tabella~\ref{tab:test-iid1-p} sono mostrati i parametri dei test eseguiti con il pattern di ritardo Latenza (descritto nella sezione precedente). Oltre al pattern Latenza abbiamo eseguito i test con il pattern Random seguendo le stesse probabilità $p$.

\begin{table}
    \centering
    \begin{tabular}{|c|c|c|c|c|c|} \hline 
         L/p&  0&  0.1&  0.2& 0.4  & 0.6\\ \hline 
         0&  L 0, p 0&  L 0, p 0.1&  L 0, p 0.2& L 0, p 0.4 &L 0, p 0.6\\ \hline 
         2&  L 2, p 0&  L 2, p 0.1&  L 2, p 0.2& L 2, p 0.4 &L 2, p 0.6\\ \hline 
         4&  L 4, p 0&  L 4, p 0.1&  L 4, p 0.2& L 4, p 0.4 &L 4, p 0.6\\ \hline 
         8&  L 8, p 0&  L 8, p 0.1&  L 8, p 0.2& L 8, p 0.4 &L 8, p 0.6\\ \hline
    \end{tabular}
    \caption{Parametri ritardo di tipo Latenza test su dati IID}
    \label{tab:test-iid1-p}
\end{table}

Essendo i risultati molto omogenei tra di loro, in Figura~\ref{fig:t00_MNIST_stag_even} sono presentati i risultati più rilevanti. Dai risultati si deduce il fatto che permettere l'aggregazione di aggiornamenti obsoleti con dati i.i.d non inficia sull'accuratezza del modello finale. Già dal round $50$ il delta è così vicino a $0$ che diventa trascurabile. Per delta si intende la differenza tra l' accuracy del modello allenato in un contesto privo di ritardi e l' accuracy del modello allenato con un pattern di ritardo specifico.

\begin{figure}[h]
\centering
\includegraphics[width=0.65\textwidth]{assets/test00_MNIST_strugglers_even2.png}
\caption{Delta accuracy del modello al proseguire dei round di allenamento rispetto ad un allenamento senza stragglers con dati i.i.d. \\
$$delta(round) = accuracy_{no\_strag}(round) - accuracy_{strag}(round) $$}\label{fig:t00_MNIST_stag_even}
\end{figure}


\subsubsection*{Stragglers con dati non uniformemente distribuiti}
Cosa accade però se la distribuzione relativa al tipo di dati non è uniforme? Abbiamo ripetuto l'esperimento con una distribuzione non uniforme delle labels in ogni dataset locale (la distribuzione delle labels è mostrata in Figura~\ref{fig:uneven_distr}).
La distribuzione dei campioni, come di vede, è molto verticale su una singola classe, volevamo sperimentare il caso limite per vedere quanto questa variabile possa potenzialmente incidere sulla convergenza del modello al variare del numero di stragglers. Come si vede nella Figura~\ref{fig:t00_acc} la convergenza finale non risente della distribuzione non uniforme della tipologia di campioni nei dataset locali. Nella Figura~\ref{fig:t00_MNIST_strugglers_uneven} si vede inoltre che la presenza di strugglers non causa grossi problemi o meglio, non ne causa affatto, infatti gli esperimenti eseguiti con i vari pattern di ritardo mostrano con una quantità ragionevole di scarto lo stesso andamento del caso con nessun ritardo.

\begin{figure}[h!]
    \centering
    \subfloat{{\includegraphics[width=0.45\textwidth]{assets/uneven_distr0.png} }}
    \qquad
    \subfloat{{\includegraphics[width=0.45\textwidth]{assets/uneven_distr1.png} }}
    \caption{Distribuzione delle labels non uniformi. Sull'asse $x$ è indicata la classe (label) e sull'asse $y$ la porzione del dataset locale formata dai campioni di quella classe. \\
    Per chiarezza sulla sinistra è mostrato il caso di un solo Client, il quale ha un "bias" verso la classe 5. Nella figura di destra sono mostrati invece tutti i Client. Anche se i Client utilizzati nei nostri test sono $100$ esistono solamente $10$ tipologie di distribuzione dei campioni, in quanto il dataset MNIST è costituito da 10 classi. Ogni tipologia di dataset sarà quindi utilizzata da $\lceil Nclient/Nclassi \rceil$}
    \label{fig:uneven_distr}
\end{figure}


\begin{figure}[h!]
\centering
\includegraphics[width=0.6\textwidth]{assets/test00_accuracy2.png}
\caption{Accuracy dei modelli allenati in assenza di ritardi. Come si nota la distribuzione delle label non inficia né sulle prestazioni del modello finale né sulla velocità di convergenza.}\label{fig:t00_acc}
\end{figure}

\begin{figure}[h!]
\centering
\includegraphics[width=0.65\textwidth]{assets/test00_MNIST_strugglers_uneven.png}
\caption{Delta accuracy del modello al proseguire dei round di allenamento rispetto ad un allenamento senza stragglers con dati non uniformemente distribuiti. $$delta(round) = accuracy_{no\_strag}(round) - accuracy_{strag}(round) $$}\label{fig:t00_MNIST_strugglers_uneven}
\end{figure}



\subsubsection{Stragglers in caso non i.i.d}
Fino ad ora abbiamo testato solamente un caso di eterogeneità statistica: la distribuzione dei dati. Cosa succede se i modelli locali non sono gli stessi, ovvero, se non esiste una funzione obiettivo globale univoca?

Per testare come cambia la convergenza quando i modelli locali sono eterogenei tra di loro, abbiamo replicato gli esperimenti proposti in \cite{DBLP:journals/corr/abs-1812-06127}. Questi esperimenti prevedono l'uso del Federated Learning per allenare una regressione logistica (Logistic Regression) su dataset sintetici.

I dataset vengono generati a partire da due parametri $\alpha$ e $\beta$ dove il primo controlla quanto i modelli locali siano diversi tra di loro ed il secondo quanto i dati sono distribuiti in modo non uniforme (viene aggiunto il caso i.i.d come benchmark di riferimento). 

Per generare questi dataset ci siamo avvalsi dell'artefatto software fornito dagli autori \footnote{https://github.com/litian96/FedProx}, utilizzando gli stessi seed per i generatori numerici pseudo casuali in modo da ottenere esattamente gli stessi dataset utilizzati.

Per prima cosa come benchmark eseguiamo il test rimuovendo i ritardi e quindi la presenza di Client stragglers. Dalla Figura~\ref{fig:non-iid} Si nota in modo evidente quanto la stabilità del modello globale sia compromessa nei casi non i.i.d e come sia proporzionalmente instabile all'aumentare dell'eterogeneità statistica. Basta però utilizzare parametri più ragionevoli come un batch size più grande o un learning rate più basso per riportare la stabilità del modello. Un ulteriore opzione è quella di utilizzare ottimizzazioni del classico algoritmo FedAvg come FedProx che risolve i problemi di instabilità come mostrato nell'articolo citato. Queste soluzioni sono efficienti e permettono di poter assicurare in pratica la convergenza del modello globale anche in situazioni eterogenee. \\

\begin{figure}[h!]
    \centering
    \subfloat{{\includegraphics[width=0.5\textwidth]{assets/drp=no_bs=10_lr=0.01.png} }}
    \subfloat{{\includegraphics[width=0.5\textwidth]{assets/drp=no_bs=50_lr=0.01.png} }}

    
   \subfloat{{\includegraphics[width=0.5\textwidth]{assets/drp=no_bs=50_lr=0.001.png} }}
    
    \caption{Risultati di FedAvg con dati non i.d.d, sulla sinistra con batch size 10 mentre sulla destra con batch size 50, nella figura sottostante sono riportati i risultati nel caso di learning rate 0.001 con batch size 50}
    \label{fig:non-iid}
\end{figure}

Cosa succede se reintroduciamo la presenza degli stragglers?
Abbiamo ripetuto gli esperimenti in due configurazioni:
\begin{itemize}
    \item Scarto di aggiornamenti obsoleti: Ogni qualvolta che un Client è in ritardo con il proprio aggiornamento il risultato del suo allenamento viene scartato. Per quanto riguarda il pattern di ritardo questi esperimenti utilizzano il Campionamento. I risultati sono mostrati in Figura~\ref{fig:non-iid-strugglers}
    \item Aggregazione degli aggiornamenti obsoleti: Quando arriva un aggiornamento di un Client in ritardo si aggrega il risultato obsoleto insieme ad i risultati in linea con il modello globale. Per quanto riguarda il pattern di ritardo questi esperimenti utilizzano lo schema Random. In Figura~\ref{fig:non-iid-strugglers-nodrop} vengono mostrati i risultati di FedAvg mentre in Figura~\ref{fig:non-iid-fedprox} quelli di FedProx.
\end{itemize}

\begin{figure}[h!]
    \centering
\subfloat{{\includegraphics[ width=0.5\textwidth]{assets/drp=.5_bs=50_lr=0.01.png} }}
   \subfloat{{\includegraphics[width=0.5\textwidth]{assets/drp=.5_bs=50_lr=0.001.png} }}
    
    \caption{Risultati di FedAvg con presenza del 50\% di strugglers ad ogni round con dati non i.d.d e drop dei Client in ritardo. Sulla sinistra sono riportati i risultati in caso di learning rate 0.01 mentre a destra i casi con learning rate 0.001}
    \label{fig:non-iid-strugglers}
\end{figure}

\begin{figure}[h!]
    \centering
    \subfloat{{\includegraphics[width=0.5\textwidth]{assets/strag=.4_bs=50_lr=0.01.png} }}
    \subfloat{{\includegraphics[width=0.5\textwidth]{assets/strag=.4_bs=50_lr=0.001.png} }}
    
    \caption{Risultati di FedAvg con presenza del 40\% di strugglers con il pattern di ritardo Random con dati non i.d.d. Sulla sinistra sono riportati i risultati in caso di learning rate 0.01 mentre a destra i casi con learning rate 0.001}
    \label{fig:non-iid-strugglers-nodrop}
\end{figure}

\begin{figure}[h!]
    \centering
    \subfloat{{\includegraphics[width=0.5\textwidth]{assets/strag=.4_bs=10_lr=0.01_mu=1.png} }}
    \subfloat{{\includegraphics[width=0.5\textwidth]{assets/strag=.4_bs=50_lr=0.001_mu=1.png} }}

  \caption{Risultati di FedProx con presenza del 40\% di strugglers utilizzando pattern di ritardo Random con dati non i.d.d. Sulla sinistra sono riportati i risultati in caso di learning rate 0.01 e batch size 10 mentre a destra i casi con learning rate 0.001 e batch size 50}
    \label{fig:non-iid-fedprox}
\end{figure}

Nel caso di scarto degli aggiornamenti obsoleti si nota una maggiore instabilità della convergenza del modello globale rispetto all'assenza di stragglers (Figura~\ref{fig:non-iid}), ma utilizzando parametri di allenamento più moderati come un learning rate più basso si risolve il problema. Nel caso  di aggregazione degli aggiornamenti obsoleti, invece, si ha una maggior instabilità rispetto alla tecnica dello scarto; in questo caso adattare i parametri di addestramento non risulta sufficiente, mentre utilizzare tecniche come FedProx risolve il problema di instabilità. Infatti come mostrato in Figura~\ref{fig:non-iid-fedprox} si nota quanto FedProx sia efficace in questi casi, rimanendo stabile in casi di parametrizzazioni non ottimali. Si noti inoltre come il caso i.i.d rallenti notevolmente la convergenza in caso di forte presenza di stragglers e in caso di utilizzo di learning rate molto limitati.

\subsubsection*{Stragglers con parametri di allenamento eterogenei}
Fino ad ora abbiamo posto l'assunto che ogni Client esegua la stessa quantità di lavoro globale, ed abbiamo notato che in questa configurazione, sebbene la presenza di stragglers infici sulla stabilità del modello, questa è facilmente gestibile con piccoli accorgimenti nella configurazione dei parametri di apprendimento. 

Cosa succede se rilassiamo questa assunzione e permettiamo ad i Client di eseguire una quantità variabile di lavoro? 

Per permettere questo esperimento abbiamo rimosso gli stragglers ed abbiamo permesso ad ogni Client di eseguire una quantità variabile di lavoro in termine di epoche. Così facendo abbiamo rimosso il fenomeno degli stragglers permettendo a questi Client di eseguire meno lavoro per non essere in ritardo. Come mostrato nell'articolo~\cite{DBLP:journals/corr/abs-1812-06127}, e come replicato con i nostri test in Figura~\ref{fig:var_ephocs} FedAvg in condizione non i.i.d tende a perdere stabilità nella convergenza e presenta il problema del' inconsistenza dell'obiettivo spigato in dettaglio nella sezione Eterogeneità~dei~Sistemi~di~Calcolo~\ref{chap:sys-eter}. FedProx in questo caso però, aggiungendo un parametro di prossimità e quindi limitando gli effetti di aggiornamenti che seguono direzioni diverse, riesce a mascherare efficacemente il problema dell'eterogeneità di sistema. FedProx non è l'unica ottimizzazione possibile infatti esistono soluzioni come FedNova~\cite{DBLP:journals/corr/abs-2007-07481}, FedAvgM~\cite{DBLP:journals/corr/abs-1909-06335} ed HySync~\cite{9407951} che utilizzano approcci differenti ma mirati al miglioramento della stabilità della convergenza del modello e in modo specifico alla mitigazione del problema di inconsistenza dell'obiettivo.

\begin{figure}
    \centering
    \subfloat{{\includegraphics[width=0.8\textwidth]{assets/var_ephocs_iid.png} }} \\
    \hspace{0.45cm}
    \subfloat{{\includegraphics[width=0.75\textwidth]{assets/var_ephocs_non-iid.png} }}
    \caption{Training loss di FedAvg e FedProx quando i Client eseguono una quantità di epoche variabili.}
    \label{fig:var_ephocs}
\end{figure}
 \section{Problemi di eterogeneità in contesti asincroni}
Avendo approfondito gli effetti dell'eterogeneità in contesti sincroni, abbiamo acquisito le conoscenze e i benchmark necessari per valutare in maniera sistematica l'approccio asincrono. Il passo successivo consiste nel definire un metodo che consenta di simulare l'esecuzione asincrona dei Client in un ambiente controllato e ripetibile. Questo approccio è fondamentale per analizzare le dinamiche asincrone, identificare eventuali criticità e ottimizzare le strategie di aggregazione e gestione dei Client, tenendo conto delle loro caratteristiche eterogenee.


\subsubsection*{Modellazione dell'aggiornamento asincrono e selezione dei Client}
Per ottenere un comportamento realistico dell' invio degli aggiornamenti abbiamo utilizzato un generatore numerico pseudo casuale con un seme costante. Essendo l'apprendimento totalmente asincrono il concetto di round decade. Definiamo l'iterazione della nostra simulazione \textit{tick}. Ogni tick rappresenta in maniera astratta un lasso di tempo variabile, poiché simula il passare del tempo tra l'invio dei modelli da parte dei client e questo è arbitrariamente non uniforme. Allo scopo dei test gli intervalli di tempo che occorrono tra un aggiornamento e l' altro non ci interessano. Poiché non esiste più il concetto di round sull'asse delle $x$ indichiamo le versioni del modello globale (indicate da $0$ a $n$: dalla più vecchia alla più nuova). 

Ad ogni tick viene selezionato in modo casuale un Client che deve inviare un aggiornamento. Il Client selezionato esegue $e$ epoche di allenamento che, per lo scopo di questo esperimento, sono un numero casuale tra ($[5,20]$)\footnote{Epoche inferiori a 5 potrebbero risultare insufficienti a consentire al modello di apprendere in modo significativo dai dati locali del client, portando a un aggiornamento poco utile o addirittura negativo per il modello globale. D'altra parte, eseguire più di 20 epoche potrebbe non essere necessario, poiché l'apprendimento federato si basa su aggiornamenti frequenti da parte dei client, e un numero elevato di epoche per ogni aggiornamento potrebbe portare a un sovraccarico computazionale eccessivo, senza apportare miglioramenti sostanziali al modello globale. }. Inoltre il client selezionato allena il modello basato su $l$ versioni precedenti del modello globale (dove $l$ può essere $0$, quindi l' ultimo modello), in modo da simulare l'effetto degli aggiornamenti obsoleti. Il parametro $l$ è legato ad una distribuzione geometrica di parametro $p$ (distribuzione descritta dalla formula $P(l=n) = p^n$). Per l'esecuzione di questo esperimento abbiamo fissato $p=0.8$ in quanto una probabilità di ritardo molto alta che ci permette di studiare il caso più estremo.

Per motivi di complessità computazionale e necessità di confronto con la fase precedente abbiamo utilizzato i dataset sintetici generati come descritto nella sezione antecedente. L'utilizzo di questi dataset sintetici ci permette inoltre di avere il controllo sulle variabili di eterogeneità statistica.

\subsubsection*{Problemi di instabilità e soluzioni}
Come primo esperimento abbiamo valutato il caso di aggregazione puramente asincrona: Ogni volta che viene ricevuto un aggiornamento da parte dell' aggregatore questo genera subito il nuovo modello globale. In Figura~\ref{fig:pure-async}
sono mostrati i risultati. Come si nota in caso di dati i.i.d non si riscontrano problemi ma, nel momento in cui si introduce un margine di eterogeneità la convergenza del modello diventa totalmente instabile.

\begin{figure}[h]
\centering
\includegraphics[width=0.8\textwidth]{assets/async_e=(8,20)_p=0.8_bs=50_lr=0.01.png}
\caption{Accuracy allenamento totalmente asincrono.}\label{fig:pure-async}
\end{figure}

\noindent Per compensare questa instabilità abbiamo introdotto la tecnica di FedBuff \cite{DBLP:journals/corr/abs-2106-06639},  la quale consiste nel "bufferizzare" gli aggiornamenti. Appena arriva un aggiornamento lo si mette nel buffer, se ci sono almeno $N$ aggiornamenti pendenti si genera il nuovo modello altrimenti si aspetta per nuovi aggiornamenti.
Il framework proposto supporta questa operazione nativamente, infatti, per motivi di performance, gli aggiornamenti vengono automaticamente messi in una coda. Abbiamo dunque ripetuto l'esperimento applicando FedBuff  con $N=10$. Come si vede in Figura~\ref{fig:fed-buff}, dove sono mostrati i risultati, questa ottimizzazione è decisamente efficiente e permette di risolvere i problemi di instabilità. Nel caso di maggior eterogeneità ($\alpha=1, \beta=1$) si notato dei picchi negativi sull' accuracy del modello, questi sono probabilmente legati problema dell'inconsistenza dell'obiettivo: ad una nuova versione collaborano modelli che presentano tutti un forte "bias", questo fa si che il modello globale si sposti verso il "bias" e che quindi si allontani dalla funzione obiettivo globale.

\begin{figure}[h!]
\centering
\includegraphics[width=0.8\textwidth]{assets/fedbuff(10)_e=(8,20)_p=0.8_bs=50_lr=0.01.png}
\caption{Accuracy allenamento asincrono con FedBuff $N=10$}\label{fig:fed-buff}
\end{figure}

\subsubsection*{Combinazione di FedBuff e FedProx }
Sebbene FedBuff sia già potenzialmente sufficiente per risolvere i problemi di instabilità, è possibile combinare questa tecnica con altre tecniche come FedProx. Quindi abbiamo ripetuto l'esperimento mantenendo $N=10$ ed integrando FedProx con parametro $mu=1$. Come si vede dalla Figura~\ref{fig:fed-buff2} le piccole instabilità che si erano ottenute con l'applicazione del semplice FedBuff sono sparite. Come effetto collaterale però abbiamo un notevole rallentamento della convergenza in caso di dati i.i.d, questo risultato è in linea con le aspettative infatti questo fenomeno è evidenziato anche nell'articolo in cui viene proposto FedProx~\cite{DBLP:journals/corr/abs-1812-06127}

\begin{figure}[h!]
\centering
\includegraphics[width=0.8\textwidth]{assets/fedbuff(10)_mu=1_e=(8,20)_p=0.8_bs=50_lr=0.01.png}
\caption{Accuracy allenamento asincrono con FedBuff $N=10$ e FedProx $mu=1$}\label{fig:fed-buff2}
\end{figure}

 \section{Framework benchmarks}
Per valutare le prestazioni del framework implementato in questa tesi abbiamo eseguito dei benchmark. Come metrica di riferimento abbiamo usato il numero di operazioni al secondo (OPS) che il sistema riesce a gestire per valutare le capacità di throughput del server. Abbiamo creato due tipi di operazioni che i Client di test possono eseguire: 1) la prima consiste nel connettersi al server, autenticarsi e scaricare l'ultima versione del modello; 2) la seconda nel connettersi, autenticarsi ed inviare un aggiornamento locale. Ogni Client eseguirà in modo alternato le due operazioni. 
Per rendere il test rappresentativo di un contesto di Federated Learning reale abbiamo ritenuto ragionevole considerare che ogni dispositivo federato impieghi un tempo nell'ordine di minuti per allenare il modello locale. È quindi molto probabile che calcolare un nuovo modello ogni secondo o frazione di secondo non sia utile ai fini dell'apprendimento federato. Per cui abbiamo configurato il server centrale in modo che aggreghi i risultati ogni $10$ secondi.

I test sono stati eseguiti con vari gradi di parallelismo per valutare la scalabilità del framework: $1, 2, 4, 8$ threads, tenendo conto che l'aggregatore risiede sempre su un thread separato. Ogni test è stato ripetuto $10$ volte ed i risultati finali sono la media delle prove ripetute.

\begin{figure}[h!]
\centering
\includegraphics[width=0.8\textwidth]{assets/bench.png}
\caption{Benchmark del framework in termini di operazioni al secondo gestibili al variare del grado di parallelismo. La linea indicata l'efficienza dell'utilizzo delle risorse espressa in percentuale, ad indicare quanto il sistema sia in grado di utilizzare le risorse disponibili al variare del numero di thread utilizzati.}\label{fig:bench}
\end{figure}


In Figura~\ref{fig:bench} vengono presentati i risultati ottenuti. Come si nota già con $4$ thread il sistema è in grado di gestire circa $100.000$ OPS. Se si suppone che un Client sia in grado di generare un aggiornamento ogni ora, il sistema è in grado di gestire un numero di dispositivi dell'ordine del milione. 
Le prestazioni del framework smettono di scalare già con $4$ threads. Questo comportamento è dovuto alla saturazione delle risorse di rete del cluster su cui sono stati eseguiti i test. Dal punto di vista teorico, il server è progettato per scalare in modo lineare con il numero di thread, grazie al design parallelizzato e al thread separato dedicato all'aggregazione. Tuttavia, le limitazioni delle risorse hardware disponibili, in particolare quelle legate alla larghezza di banda e alla latenza della rete, hanno rappresentato un collo di bottiglia per le prestazioni complessive.
Pertanto, i risultati presentati vanno ritenuti parziali e rappresentativi solo delle capacità del framework all'interno del contesto specifico di test. Come sviluppo futuro, prevediamo di valutare le prestazioni del framework su reti di interconnessione a più larga banda.






 \section{Considerazioni}
Sebbene l'eterogeneità statistica e di sistema pongano delle problematiche a livello di convergenza del modello, nella prima sezione di sperimentazione abbiamo mostrato empiricamente come queste, in contesti sincroni, sono facilmente gestibili utilizzando configurazioni dei parametri di allenamento appropriate e tecniche di ottimizzazione dove necessario. \\

In contesti asincroni invece, rispetto a quelli sincroni, l'eterogeneità statistica e di sistema creano problemi di instabilità maggiori. Utilizzando un approccio completamente asincrono (Dove si genera una nuova versione del modello globale ad ogni aggiornamento ricevuto) abbiamo notato il modello divergere in pratica nei casi in cui venga meno la premessa di dati i.i.d. Tuttavia abbiamo notato miglioramenti significativi nell'utilizzare approcci ibridi come FedBuff \cite{DBLP:journals/corr/abs-2106-06639} che permettono di stabilizzare notevolmente la convergenza del modello. 

Nonostante i notevoli miglioramenti ottenuti con FedBuff  \cite{DBLP:journals/corr/abs-2106-06639} in casi di alta eterogeneità rimangono, anche se attenuati, problemi di stabilità del modello legati al fenomeno dell'inconsistenza dell' obiettivo \cite{DBLP:journals/corr/abs-2007-07481} (problematica discussa in dettaglio nel capitolo di Background~\ref{background}). 

Per risolvere il problema dell'instabilità del modello in contesti altamente eterogenei abbiamo combinato FedBuff \cite{DBLP:journals/corr/abs-2106-06639} con FedProx \cite{DBLP:journals/corr/abs-1812-06127}. Questa combinazione ha permesso di eliminare completamente il problema dell'inconsistenza dell'obiettivo. \\

Il lavoro svolto dimostra che, sebbene la gestione dell'eterogeneità in contesti sincroni e asincroni possa sembrare una sfida complessa, l'adozione di strategie appropriate consente di affrontare e risolvere efficacemente questi problemi, garantendo la stabilità e la convergenza del modello globale. \\

Oltre ad i risultati relativi alla convergenza del modello globale questi esperimenti hanno dimostrato che il framework\footnote{https://github.com/mamodev/Async-Federated-Learnig} sviluppato si è dimostrato in grado di riprodurre risultati presentati in articoli scientifici utilizzando simulatori, come \cite{DBLP:journals/corr/abs-1812-06127} e \cite{DBLP:journals/corr/abs-2106-06639}, in ambienti reali e fisicamente distribuiti. Il framework si è rivelato sufficientemente flessibile per introdurre più tecniche e verificarne la bontà (anche in contesti sincroni che rimangono fuori dal scopo principale del software stesso). La sua architettura modulare ha facilitato l'integrazione delle varie strategie, in particolare quella di FedBuff, rendendo la sua implementazione semplice ed efficace. Infatti, la bufferizzazione degli aggiornamenti è già una funzionalità nativa del framework, progettata per ottimizzare le performance. Il framework oltre ad una notevole flessibilità ha dimostrato di essere in grado di gestire grossi carichi di lavoro: nell'ordine di centinaia di migliaia di OPS. 


\clearpage{}
\clearpage{}\chapter{Conclusioni e Sviluppi futuri}\label{chap:conclusioni}
Negli ultimi anni, il Federated Learning (FL) ha mostrato un grande potenziale per superare i limiti del Machine Learning tradizionale, offrendo soluzioni innovative in termini di privacy dei dati e scalabilità. Tuttavia, la sua applicazione pratica, specialmente in contesti con dispositivi eterogenei, presenta sfide significative legate alle diverse capacità dei dispositivi, alla complessità delle comunicazioni e alla gestione distribuita. In questa tesi, abbiamo affrontato tali problematiche proponendo un framework\footnote{\label{nt:framework}Codice sorgente: https://github.com/mamodev/Async-Federated-Learnig} progettato per semplificare l’ adozione del Federated Learning asincrono, consentendo agli sviluppatori di concentrarsi su aspetti chiave come la definizione dei modelli e l’ analisi dei dati, senza doversi occupare delle complessità tecniche.

Nel corso di questo elaborato abbiamo dapprima introdotto il Federated Learning, le sue sfide (come l'eterogeneità dei dati e dei sistemi di calcolo) ed alcune tra le possibili architetture. Dopo aver dato un quadro completo del background necessario, abbiamo analizzato le soluzioni attualmente presenti sul mercato, valutandone punti di forza e debolezza, ed evidenziato una mancanza di opzioni appropriate per contesti cross-device ed altamente eterogenei su grossa scala (centinaia di migliaia di Client). Dopo aver chiarito le motivazioni che hanno guidato lo sviluppo del framework\footnoteref{nt:framework} oggetto di questa tesi abbiamo descritto il design e l'architettura software in dettaglio. Infine abbiamo condotto un'ampia sperimentazione per validare il funzionamento stesso del software, riproponendo con successo esperimenti presenti in articoli accademici come \cite{DBLP:journals/corr/abs-1812-06127}. Abbiamo inoltre dimostrato empiricamente  che, sebbene la gestione dell'eterogeneità in contesti sincroni e asincroni possa sembrare una sfida complessa, l’ adozione di strategie appropriate consente di affrontare e risolvere efficacemente questi problemi, garantendo la stabilità e la convergenza del modello globale. Il framework\footnoteref{nt:framework} sviluppato si distingue per la sua attenzione all'efficienza e alla flessibilità. La scelta di ottimizzare le prestazioni, riducendo il carico computazionale e migliorando la gestione delle risorse, consente una scalabilità superiore rispetto a molte soluzioni esistenti che abbiamo analizzato. Anche la comunicazione tra dispositivi è stata ottimizzata per garantire una maggiore fluidità e reattività del sistema, permettendo di gestire efficacemente numerosi dispositivi connessi in simultanea. Inoltre, il framework\footnoteref{nt:framework} offre possibilità di personalizzazione, consentendo di adattare le strategie di aggregazione alle esigenze specifiche degli sviluppatori.I test sperimentali hanno dimostrato come questo framework\footnoteref{nt:framework} renda triviale lo sviluppo di applicazioni di Federated Learning asincrono, affrontando con successo problemi di eterogeneità e scalabilità tipici dei dispositivi distribuiti.

\subsubsection*{Sviluppi futuri}
Il framework\footnoteref{nt:framework} sviluppato offre numerose possibilità di evoluzione per affrontare in modo ancora più efficace le sfide legate al Federated Learning. Una direzione promettente è il passaggio verso un'architettura più decentralizzata e resiliente, con l'integrazione di meccanismi per migliorare l'affidabilità e la scalabilità del sistema. L'adozione di tecniche avanzate per ottimizzare le comunicazioni e ridurre il consumo di banda rappresenta un altro potenziale ambito di miglioramento.\\
Un'ulteriore area di sviluppo riguarda l'integrazione di strumenti avanzati per la protezione della privacy, come la crittografia avanzata o la privacy differenziale, che offrono garanzie ancora più solide sul trattamento sicuro dei dati. Inoltre, l'estensione del supporto a una gamma più ampia di modelli di learning, inclusi modelli non supervisionati e di apprendimento per rinforzo, potrebbe rendere il framework\footnoteref{nt:framework} ancora più versatile e applicabile in contesti diversi.\\
Infine, migliorare l’ accessibilità attraverso documentazione esaustiva e interfacce intuitive potrebbe ampliare il pubblico di potenziali utenti. L'efficacia del framework\footnoteref{nt:framework} potrebbe essere ulteriormente validata attraverso test su larga scala in collaborazione con realtà industriali o accademiche, per dimostrare la sua capacità di operare con un elevato numero di dispositivi in ambienti reali.

\subsubsection*{Considerazioni finali}
Ritengo che il framework proposto, disponile su github\footnoteref{nt:framework}, rappresenti un passo avanti verso un Federated Learning più accessibile anche agli sviluppatori non esperti in infrastrutture e sistemi distribuiti. Nella sua fase di design e sviluppo abbiamo affrontato con efficacia le sfide legate all'eterogeneità e alla scalabilità, fornendo agli sviluppatori una soluzione flessibile e intuitiva per implementare applicazioni che rispettano la privacy e sfruttano al meglio le risorse di dispositivi eterogenei. Con i potenziali sviluppi futuri descritti nella sezione precedente, l'obiettivo è ampliare ulteriormente le capacità tecnologiche e di utilizzo del framework sviluppato, rendendolo un elemento importante per l'innovazione nell'ambito del Federated Learning, promuovendo la collaborazione, la sicurezza e l'efficienza in ambienti distribuiti sempre più complessi.
\clearpage{}

\bibliographystyle{alpha}
\bibliography{cite}

\clearpage{}\clearpage{}

\end{document}
